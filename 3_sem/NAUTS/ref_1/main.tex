\documentclass[a4paper,14pt]{extarticle} % Используем extarticle для поддержки шрифта 14
\usepackage[T2A]{fontenc}
\usepackage[utf8]{inputenc}
\usepackage[russian]{babel}
\usepackage{amsmath}
\usepackage{amssymb}
\usepackage{amsfonts}
\usepackage{graphicx}
\usepackage{geometry}
\usepackage{hyperref}
\usepackage{setspace} % Пакет для изменения межстрочного интервала
\geometry{top=2cm, bottom=2cm, left=3cm, right=1.5cm}
\onehalfspacing % Установка межстрочного интервала 1.5



\begin{document}

\begin{titlepage}
    \begin{center}
        \large
        Министерство науки и высшего образования Российской Федерации \\
        Санкт-Петербургский государственный электротехнический университет \\
        \textbf{«ЛЭТИ» им. В.И. Ульянова (Ленина)} \\
        Кафедра систем автоматического управления

        \vfill

        \textbf{Реферат} \\
        по дисциплине \\
        \textbf{«Нелинейное адаптивное управление в технических системах»}

        \vfill

        Студент группы 9492 \hfill Викторов А.А. \\
        Преподаватель \hfill Путов В.В.

        \vfill
        Санкт-Петербург \\
        2024
    \end{center}
\end{titlepage}

\tableofcontents
\newpage

\section{Введение}
Теория нелинейного адаптивного управления в технических системах представляет собой одну из наиболее актуальных и динамично развивающихся областей современной теории управления. Актуальность этой темы обусловлена тем, что большинство реальных систем в природе и технике являются нелинейными, а их параметры могут изменяться во времени или быть неизвестными. 

Примером нелинейных систем служат современные автотранспортные средства, оснащенные системами стабилизации, которые динамически подстраивают управление тормозами и рулевым управлением в зависимости от условий движения, таких как поверхность дороги и скорость автомобиля. В случае автономных транспортных средств (беспилотных автомобилей) требуется ещё более точная система адаптивного управления для обеспечения безопасности на дороге. Эти системы должны учитывать множество внешних факторов: изменение погодных условий, препятствия на дороге и даже стиль вождения других участников движения.

Основное преимущество адаптивных систем управления заключается в том, что они могут автоматически подстраиваться под изменяющиеся условия среды, не требуя вмешательства человека. Благодаря этому такие системы находят широкое применение не только в автомобильной, но и в авиационной и космической технике. Например, в системах управления полётом самолётов или беспилотных летательных аппаратов адаптивное управление позволяет учитывать изменяющиеся внешние условия, такие как изменение атмосферного давления, турбулентность, и корректировать траекторию полёта.

Таким образом, адаптивное управление в технических системах играет ключевую роль в обеспечении устойчивости и безопасности современных высокотехнологичных решений, от автомобилей до спутников.

\newpage

\section{Основные понятия теории обыкновенных дифференциальных уравнений (ОДУ)}

\subsection{Определение ОДУ и его решения}
Обыкновенное дифференциальное уравнение (ОДУ) — это уравнение, которое связывает функцию и её производные по одной независимой переменной (обычно времени). Общий вид ОДУ можно записать как:
\[
\frac{dx}{dt} = f(x, t),
\]
где \(x \in \mathbb{R}^n\) — это вектор состояния системы, \(t\) — независимая переменная (время), а функция \(f(x, t)\) описывает законы изменения параметров системы. Решением данного уравнения является функция \(x(t)\), которая показывает, как состояние системы изменяется с течением времени.

Пример: для описания движения автомобиля можно использовать дифференциальные уравнения, где \(x(t)\) — это положение и скорость автомобиля в определённый момент времени, а \(f(x, t)\) — это сила, приложенная к автомобилю, например, сила тяги двигателя или сопротивление воздуха.

\subsection{Начальные условия и задача Коши}
Чтобы решить ОДУ, необходимо задать начальные условия, то есть знать состояние системы в начальный момент времени. Задача Коши формулируется как:
\[
\frac{dx}{dt} = f(x, t), \quad x(t_0) = x_0,
\]
где \(x_0\) — начальное состояние системы в момент времени \(t_0\). Например, если мы моделируем полёт ракеты, то в начальный момент времени задаются её начальная скорость и положение. Решение задачи Коши позволяет определить поведение ракеты в будущем на основе этих данных.

\subsection{Фазовое пространство и фазовые траектории}
Фазовое пространство — это математическое пространство, каждая точка которого соответствует определённому состоянию системы. Например, если система описывается двумя переменными, положение и скорость, то каждая точка фазового пространства будет представлять определённое состояние (комбинацию положения и скорости).

Фазовая траектория — это кривая в фазовом пространстве, которая показывает, как изменяется состояние системы со временем. В контексте управления, фазовые траектории используются для визуализации поведения систем и анализа их устойчивости.

\subsection{Автономные и неавтономные системы}
ОДУ называется автономным, если оно не зависит явно от времени, то есть его правая часть не содержит переменной \(t\):
\[
\frac{dx}{dt} = f(x).
\]
Примером автономных систем являются многие механические системы, где законы движения определяются только состоянием системы (например, положением и скоростью), но не временем.

Неавтономные системы, напротив, зависят от времени:
\[
\frac{dx}{dt} = f(x, t).
\]
Примером неавтономной системы является автомобиль, движущийся по дороге, когда его ускорение зависит как от текущей скорости, так и от времени, поскольку условия движения могут изменяться (например, появление светофора).

\subsection{Линейные и нелинейные системы}
Линейными называются системы, описываемые дифференциальными уравнениями, в которых изменения состояния пропорциональны изменениям входных параметров. Для линейной системы можно записать:
\[
\frac{dx}{dt} = A(t)x + b(t),
\]
где \(A(t)\) — это матрица коэффициентов, а \(b(t)\) — вектор свободных членов.

Однако большинство реальных систем, особенно в технике, являются нелинейными, то есть их поведение не может быть описано линейными уравнениями. Примером нелинейных систем является управление автомобилем на высокой скорости, где малые изменения угла поворота руля могут приводить к резким изменениям траектории.

Нелинейные системы сложнее анализировать, поскольку они могут обладать такими свойствами, как наличие нескольких точек равновесия, возможность возникновения хаотического поведения и т.д.

\newpage

\section{Существование и единственность решений}

\subsection{Теорема Пеано о существовании решения}
Теорема Пеано — одна из ключевых теорем в теории дифференциальных уравнений, которая даёт достаточные условия существования решения задачи Коши. Она утверждает, что если функция \( f(x, t) \) непрерывна в некоторой области \( D = \{ (x, t): ||x - x_0|| \leq a, |t - t_0| \leq b \} \), то существует решение задачи Коши:
\[
\frac{dx}{dt} = f(x, t), \quad x(t_0) = x_0,
\]
определённое на интервале \( [t_0 - h, t_0 + h] \), где \( h > 0 \). Эта теорема гарантирует существование локального решения, но не утверждает его единственность.

Важно понимать, что теорема Пеано описывает только существование решения в некотором локальном временном интервале. Это означает, что для более сложных систем, где динамика системы может кардинально меняться в зависимости от времени или состояния, решение существует, но его поведение может быть непредсказуемым, что требует дополнительных методов анализа.

Примером применения этой теоремы является задача предсказания движения объекта в динамической среде, где законы изменения состояния объекта (его положение и скорость) зависят от времени и окружающих условий. Теорема Пеано гарантирует, что решение будет существовать в некоторый промежуток времени, начиная с заданных начальных условий.

\subsection{Теорема Пикара-Линделёфа о существовании и единственности решения}
Теорема Пикара-Линделёфа даёт более сильные условия, необходимые для того, чтобы решение задачи Коши было не только существующим, но и единственным. Она утверждает, что если функция \( f(x, t) \) удовлетворяет условию Липшица по переменной \( x \) в области \( D = \{ (x, t): ||x - x_0|| \leq a, |t - t_0| \leq b \} \), то существует единственное решение задачи Коши:
\[
\frac{dx}{dt} = f(x, t), \quad x(t_0) = x_0,
\]
определённое на некотором интервале \( [t_0 - h, t_0 + h] \), где \( h > 0 \).

Условие Липшица предполагает, что существует константа \( L > 0 \), такая, что:
\[
||f(x_1, t) - f(x_2, t)|| \leq L||x_1 - x_2||,
\]
для любых \( x_1, x_2 \in D \). Это условие обеспечивает "контролируемое" поведение функции \( f \) относительно её аргументов, что исключает возможность существования нескольких решений для одних и тех же начальных условий.

Примером применения этой теоремы может быть система управления беспилотным летательным аппаратом, где необходимо точно предсказывать траекторию его полёта при известных начальных условиях. Теорема Пикара-Линделёфа гарантирует, что траектория будет единственной при соблюдении определённых ограничений на систему.

\subsection{Глобальное существование решений}
Теоремы Пеано и Пикара-Линделёфа обеспечивают лишь локальное существование решений, то есть на ограниченном временном интервале. Однако для многих практических систем важно знать, что решение существует на всём временном промежутке, то есть глобально. Для этого вводятся дополнительные условия.

Глобальное существование решений возможно, если правая часть уравнения \( f(x, t) \) ограничена по норме. Более точно, если выполняется условие:
\[
||f(x, t)|| \leq M(1 + ||x||),
\]
где \( M \) — положительная константа, то решение задачи Коши будет существовать на всём временном промежутке \( [t_0, \infty) \). Это условие говорит о том, что скорость изменения системы не может бесконечно возрастать, что предотвращает "уход" решения в бесконечность за конечное время.

Глобальное существование важно, например, при анализе устойчивости орбитальных полётов спутников. Для надёжного управления спутником нужно знать, что решение уравнений движения будет существовать и быть корректным на всём временном промежутке, пока спутник находится на орбите.

\subsection{Непродолжаемые решения}
Непродолжаемые решения — это такие решения дифференциальных уравнений, которые определены только на конечном временном интервале и не могут быть продолжены за его пределы. Это может происходить из-за особенностей системы, таких как сингулярности или приближение к точкам, где уравнение теряет физический смысл.

Примером непродолжаемого решения является система, в которой происходит "обрыв" решения, например, в случае движения объекта, который за конечное время достигает точки сингулярности (неопределённости) или сталкивается с препятствием. Это может наблюдаться, например, при моделировании движения ракеты, когда она сталкивается с поверхностью земли или другим телом.

Непродолжаемые решения имеют важное значение в исследовании систем, где поведение системы может стать непредсказуемым или некорректным за конечное время, что требует дополнительных методов анализа или корректировки модели.

\newpage
\section{Устойчивость решений}

\subsection{Устойчивость по Ляпунову}
Устойчивость по Ляпунову — это фундаментальное понятие, используемое для анализа динамических систем. Оно позволяет оценить, как система реагирует на малые возмущения её начальных условий. Важность этого анализа заключается в том, что реальная система, как правило, подвержена влиянию различных внешних факторов (шум, колебания, ошибки измерений), и устойчивость означает, что небольшие отклонения в начальных условиях не приведут к значительным изменениям поведения системы.

Пусть \( x^*(t) \) — это решение системы дифференциальных уравнений:
\[
\frac{dx}{dt} = f(x, t),
\]
и \( x^*(t) \) представляет собой равновесное состояние системы. Решение \( x^*(t) \) называется устойчивым по Ляпунову, если для любого \( \varepsilon > 0 \) существует такое \( \delta(\varepsilon) > 0 \), что для всех решений \( x(t) \) системы, начальные условия которых удовлетворяют условию:
\[
||x(t_0) - x^*(t_0)|| < \delta,
\]
выполняется неравенство:
\[
||x(t) - x^*(t)|| < \varepsilon \quad \forall t \geq t_0.
\]
Это означает, что малые отклонения от равновесного состояния в начальный момент времени приводят к малым отклонениям на всём временном интервале.

Пример: если система представляет собой маятник, то его равновесное положение — это точка, в которой он покоится. Устойчивость по Ляпунову означает, что если маятник отклонить на малую величину и отпустить, он вернётся к состоянию покоя, и его отклонение со временем не будет расти.

\subsection{Асимптотическая устойчивость}
Асимптотическая устойчивость является более сильным свойством, чем устойчивость по Ляпунову. Она гарантирует не только ограниченность отклонений от равновесного состояния, но и сходимость к нему с течением времени.


\begin{definition}
Решение $x^(t)$ называется асимптотически устойчивым, если оно устойчиво по Ляпунову и, кроме того, существует $\eta(t_0) > 0$ такое, что из $|x(t_0) - x^(t_0)| < \eta$ следует
[
\lim_{t \to \infty} |x(t) - x^*(t)| = 0
]
\end{definition}


Асимптотическая устойчивость имеет большое значение в теории управления, так как она обеспечивает возвращение системы к желаемому состоянию после воздействия возмущений.


\begin{theorem}Достаточное условие асимптотической устойчивости
Пусть для системы $\dot{x} = f(x)$ существует положительно определенная функция $V(x)$ такая, что $\dot{V}(x)$ отрицательно определена в некоторой окрестности точки $x = 0$. Тогда положение равновесия $x = 0$ асимптотически устойчиво.
\end{theorem}


Это условие часто используется при анализе нелинейных систем с помощью метода функций Ляпунова.
\subsection{Равномерная устойчивость}
В некоторых случаях важно, чтобы устойчивость системы не зависела от выбора начального момента времени. Это приводит к понятию равномерной устойчивости.
\begin{definition}[Равномерная устойчивость]
Решение $x^(t)$ называется равномерно устойчивым, если для любого $\varepsilon > 0$ существует $\delta(\varepsilon) > 0$ (не зависящее от $t_0$) такое, что из $|x(t_0) - x^(t_0)| < \delta$ следует $|x(t) - x^*(t)| < \varepsilon$ для всех $t \geq t_0 \geq 0$.
\end{definition}
Равномерная устойчивость гарантирует, что поведение системы "одинаково хорошо" независимо от момента начала наблюдения.
\section{Экспоненциальная устойчивость}

Экспоненциальная устойчивость представляет собой особую форму устойчивости, которая не только гарантирует, что решения системы останутся близкими к положению равновесия при малых возмущениях, но и предоставляет количественную характеристику скорости их сходимости. В отличие от асимптотической устойчивости, которая утверждает лишь факт сходимости, экспоненциальная устойчивость описывает процесс сходимости в виде экспоненциального закона.

\textbf{Определение:} Решение \(x^*(t)\) системы называется экспоненциально устойчивым, если существуют положительные константы \(\alpha\), \(\beta\) и \(\gamma\), такие что
\[
||x(t) - x^*(t)|| \leq \alpha ||x(t_0) - x^*(t_0)|| e^{-\beta(t - t_0)},
\]
для всех \(t \geq t_0 \geq 0\) и всех начальных условий, удовлетворяющих \(||x(t_0) - x^*(t_0)|| < \gamma\).

\subsection{Физический смысл экспоненциальной устойчивости}

Экспоненциальная устойчивость характеризует скорость, с которой траектории динамической системы стремятся к состоянию равновесия. В физическом смысле это означает, что любые отклонения от состояния равновесия будут уменьшаться с течением времени по экспоненциальному закону. Чем больше величина параметра \(\beta\), тем быстрее система возвращается к своему состоянию равновесия после возмущений. Это свойство крайне важно в задачах управления, поскольку системы с экспоненциальной устойчивостью демонстрируют быстрое восстановление после малых отклонений, что обеспечивает высокую степень надёжности и предсказуемости.

\subsection{Сравнение с другими видами устойчивости}

Экспоненциальная устойчивость является более сильным свойством, чем асимптотическая устойчивость. В то время как асимптотическая устойчивость гарантирует, что отклонения от положения равновесия будут со временем стремиться к нулю, экспоненциальная устойчивость даёт более строгие гарантии, описывая скорость этого процесса. Кроме того, она позволяет оценить степень устойчивости системы на основе величины параметра \(\beta\), что может быть использовано для количественного анализа и проектирования систем управления.

Равномерная экспоненциальная устойчивость также играет важную роль, особенно в случае неавтономных систем. Равномерная экспоненциальная устойчивость означает, что скорость сходимости к положению равновесия не зависит от начального времени \(t_0\), что особенно важно в системах, где параметры или возмущения могут изменяться со временем.

\subsection{Экспоненциальная устойчивость и системы управления}

В системах управления экспоненциальная устойчивость является желаемым свойством, поскольку она гарантирует не только стабильность, но и скорость восстановления после возмущений. Это свойство особенно важно в реальных приложениях, таких как робототехника, системы наведения и стабилизации, а также в системах автоматического регулирования. В таких системах необходимо, чтобы отклонения от нормального режима работы устранялись как можно быстрее и с гарантированной скоростью, чтобы избежать накопления ошибок или деградации работы системы.

\subsection{Методы доказательства экспоненциальной устойчивости}

Для доказательства экспоненциальной устойчивости нелинейных систем могут быть использованы различные методы, в том числе:
\begin{itemize}
    \item \textbf{Метод функций Ляпунова:} Построение подходящей функции Ляпунова, для которой существует экспоненциальная оценка сходимости, является классическим подходом к доказательству экспоненциальной устойчивости.
    \item \textbf{Метод линеаризации:} Если линеаризованная система в окрестности точки равновесия обладает экспоненциальной устойчивостью (все собственные значения матрицы Якоби имеют отрицательные вещественные части), то и исходная нелинейная система может быть экспоненциально устойчива в этой окрестности.
    \item \textbf{Метод векторных полей:} Анализ векторных полей системы может также предоставить информацию о глобальном поведении системы и подтвердить её экспоненциальную устойчивость.
\end{itemize}

\subsection{Пример экспоненциально устойчивой системы}

Рассмотрим линейную систему:
\[
\dot{x} = Ax,
\]
где \(A\) — матрица с отрицательными вещественными частями всех собственных чисел. Для этой системы можно построить квадратичную функцию Ляпунова \(V(x) = x^T P x\), где \(P\) — симметричная положительно определённая матрица, удовлетворяющая уравнению Ляпунова:
\[
A^T P + P A = -Q,
\]
где \(Q\) — произвольная положительно определённая матрица. Для такой системы можно показать, что
\[
V(x(t)) \leq V(x(0)) e^{-\lambda_{\min}(Q)t},
\]
где \(\lambda_{\min}(Q)\) — минимальное собственное значение матрицы \(Q\). Это доказывает экспоненциальную устойчивость системы.

\subsection{Применение экспоненциальной устойчивости в адаптивных системах}

Экспоненциальная устойчивость является особенно важным свойством в адаптивных системах управления, где параметры системы могут изменяться во времени. Адаптивные системы проектируются таким образом, чтобы автоматически корректировать свои параметры при изменении характеристик управляемого объекта. Для обеспечения стабильной работы таких систем необходимо, чтобы сходимость оценок параметров происходила экспоненциально быстро, что гарантирует быструю адаптацию и минимизацию ошибок. В теории адаптивного управления это свойство тесно связано с понятием персистентности возбуждения (PE-условие), которое гарантирует экспоненциальную сходимость оценок параметров.

\subsection{Преимущества экспоненциальной устойчивости}

Экспоненциальная устойчивость обладает рядом преимуществ по сравнению с другими видами устойчивости:
\begin{itemize}
    \item Гарантирует быструю сходимость решений, что важно в реальных системах с ограниченными ресурсами времени.
    \item Позволяет проводить количественный анализ устойчивости на основе временных оценок, что упрощает проектирование систем управления.
    \item Применима в задачах оптимального управления, где требуется не только стабильность, но и высокая скорость выполнения управляющих воздействий.
\end{itemize}
Таким образом, экспоненциальная устойчивость является ключевым свойством для анализа и синтеза устойчивых и надёжных систем управления.


\section{Метод линеаризации}

Метод линеаризации является одним из ключевых инструментов анализа нелинейных систем в окрестности точки равновесия. Этот метод позволяет аппроксимировать поведение нелинейной системы линейной системой в малой окрестности точки равновесия.

\subsection{Основные принципы линеаризации}

Рассмотрим нелинейную систему вида:
\[
\dot{x}(t) = f(x(t)),
\]
где \(x(t) \in \mathbb{R}^n\) — вектор состояния, \(f: \mathbb{R}^n \to \mathbb{R}^n\) — нелинейная функция. Пусть \(x^*\) — точка равновесия системы, т.е. \(f(x^*) = 0\).

Линеаризация системы в окрестности \(x^*\) осуществляется путём разложения функции \(f(x)\) в ряд Тейлора:
\[
f(x) = f(x^*) + J(x^*)(x - x^*) + O(||x - x^*||^2),
\]
где \(J(x^*)\) — матрица Якоби функции \(f\) в точке \(x^*\).

Линеаризованная система имеет вид:
\[
\dot{x}(t) = J(x^*)(x(t) - x^*).
\]

\subsection{Матрица Якоби}

Матрица Якоби \(J(x^*)\) играет ключевую роль в анализе устойчивости линеаризованной системы. Она определяется как:
\[
J(x^*) = \left[\frac{\partial f_i}{\partial x_j}\right]_{x^*},
\]
где \(\frac{\partial f_i}{\partial x_j}\) — частная производная \(i\)-й компоненты функции \(f\) по \(j\)-й переменной, вычисленная в точке \(x^*\).

\subsection{Теорема о линеаризации}

\textbf{Теорема:} Пусть \(f\) — достаточно гладкая функция, удовлетворяющая условию (2.30). Равновесное состояние \(x^*\) системы (2.1) асимптотически устойчиво, если матрица системы \(J(x^*)\) гурвицева, то есть все вещественные части собственных чисел матрицы \(J(x^*)\) отрицательны.

Эта теорема предоставляет достаточное условие для асимптотической устойчивости нелинейной системы на основе анализа её линеаризации.

\subsection{Ограничения метода линеаризации}

Важно отметить, что метод линеаризации имеет ограничения:
\begin{itemize}
    \item Он применим только в малой окрестности точки равновесия.
    \item Теорема о линеаризации даёт только достаточное условие устойчивости.
    \item Существуют системы, которые асимптотически устойчивы, но не удовлетворяют условиям теоремы о линеаризации.
\end{itemize}

\subsection{Пример применения метода линеаризации}

Рассмотрим нелинейный маятник, описываемый уравнением:
\[
\theta'' + \sin(\theta) = 0,
\]
где \(\theta\) — угол отклонения маятника от вертикали.

Линеаризация этой системы в окрестности точки равновесия \(\theta = 0\) даёт:
\[
\theta'' + \theta = 0.
\]
Эта линейная система представляет собой гармонический осциллятор, что соответствует поведению маятника при малых отклонениях.

\section{Глобальная устойчивость}

Глобальная устойчивость является более сильным свойством динамической системы по сравнению с локальной устойчивостью. Она гарантирует, что система вернётся к своему равновесному состоянию независимо от начальных условий или величины возмущений.

\subsection{Определение глобальной устойчивости}

Согласно материалам лекции, система (2.1) называется глобально асимптотически устойчивой (или асимптотически устойчивой в целом), если выполняются следующие условия:
\begin{enumerate}
    \item Система является полной на \(\mathbb{R}^n\).
    \item Система имеет единственное равновесное состояние \(x^*\).
    \item Равновесное состояние \(x^*\) является устойчивым по Ляпунову.
    \item Равновесное состояние \(x^*\) является глобально притягивающим, то есть для любого начального состояния \(x_0 \in \mathbb{R}^n\) выполняется:
    \[
    \lim_{t \to \infty} ||x(t) - x^*|| = 0,
    \]
    где \(x(t)\) — решение системы с начальным условием \(x(0) = x_0\).
\end{enumerate}

\subsection{Сравнение с локальной устойчивостью}

В отличие от локальной устойчивости, которая гарантирует возврат к равновесию только для достаточно малых отклонений, глобальная устойчивость обеспечивает сходимость к равновесию для любых начальных условий. Это значительно более сильное свойство, которое труднее доказать, но оно предоставляет более надёжные гарантии поведения системы.

\subsection{Глобальная экспоненциальная устойчивость}

Ещё более сильным свойством является глобальная экспоненциальная устойчивость.

\textbf{Определение:} Система (2.1) называется глобально экспоненциально устойчивой, если существуют положительные константы \(\alpha\) и \(\beta\), такие что для всех начальных условий \(x_0 \in \mathbb{R}^n\) и всех \(t \geq 0\) выполняется:
\[
||x(t) - x^*|| \leq \beta ||x_0 - x^*|| e^{-\alpha t}.
\]
Это определение даёт не только качественную, но и количественную характеристику скорости сходимости решений к равновесию.

\subsection{Методы анализа глобальной устойчивости}

Анализ глобальной устойчивости часто требует более сложных методов, чем анализ локальной устойчивости. Основные подходы включают:
\begin{enumerate}
    \item Метод функций Ляпунова.
    \item Метод предельных множеств.
    \item Метод сравнения.
    \item Метод векторных полей.
\end{enumerate}

\subsection{Пример глобально устойчивой системы}

Рассмотрим нелинейную систему:
\[
\frac{dx}{dt} = -x^3, \quad \frac{dy}{dt} = -y.
\]
Эта система имеет единственное равновесное состояние \((0, 0)\). Функция Ляпунова \(V(x, y) = \frac{x^2}{2} + \frac{y^2}{2}\) удовлетворяет условиям глобальной асимптотической устойчивости для этой системы:
\[
\frac{dV}{dt} = x \frac{dx}{dt} + y \frac{dy}{dt} = -x^4 - y^2 \leq 0,
\]
причём \(\frac{dV}{dt} = 0\) только в точке \((0, 0)\), что доказывает глобальную асимптотическую устойчивость системы.

\subsection{Значение глобальной устойчивости в теории управления}

Глобальная устойчивость имеет огромное значение в теории управления и практических приложениях:
\begin{itemize}
    \item Робастность.
    \item Гарантии производительности.
    \item Упрощение управления.
    \item Безопасность.
\end{itemize}


\section{Заключение}

В данной работе были рассмотрены основные понятия и методы анализа устойчивости нелинейных динамических систем, включая экспоненциальную устойчивость, метод линеаризации, глобальную устойчивость и метод функций Ляпунова. Мы обсудили, как эти методы позволяют охарактеризовать поведение системы в окрестности точек равновесия и на всём пространстве состояний.

Особое внимание было уделено экспоненциальной и глобальной устойчивости, как наиболее важным свойствам систем управления. Экспоненциальная устойчивость обеспечивает быструю сходимость решений, а глобальная устойчивость гарантирует возврат системы к равновесию независимо от начальных условий. Рассмотренные методы линеаризации и функций Ляпунова являются основными инструментами для доказательства этих свойств.

Понимание и применение этих методов играет ключевую роль в разработке и анализе систем управления, обеспечивающих надёжную и стабильную работу в условиях неопределённости и возмущений. В будущем, дальнейшее развитие теории нелинейных систем и методов устойчивости позволит улучшить управление сложными системами в различных приложениях, от автоматизации до робототехники и аэрокосмических технологий.


\end{document}