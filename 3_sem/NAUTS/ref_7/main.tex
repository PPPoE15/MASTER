\documentclass[a4paper,14pt]{extarticle} % Используем extarticle для поддержки шрифта 14
\usepackage[utf8]{inputenc}
\usepackage[T2A]{fontenc}
\usepackage[russian]{babel}
\usepackage{amsmath}
\usepackage{amssymb}
\usepackage{graphicx}
\usepackage[left=3cm,right=2cm,top=2cm,bottom=2cm]{geometry}
\linespread{1.5}

\begin{document}

\begin{titlepage}
    \begin{center}
        \large
        Министерство науки и высшего образования Российской Федерации \\
        Санкт-Петербургский государственный электротехнический университет \\
        \textbf{«ЛЭТИ» им. В.И. Ульянова (Ленина)} \\
        Кафедра систем автоматического управления

        \vfill

        \textbf{Реферат} \\
        по дисциплине \\
        \textbf{«Нелинейное адаптивное управление в технических системах»}

        \vfill

        Студент группы 9492 \hfill Викторов А.Д. \\
        Преподаватель \hfill Путов В.В.

        \vfill
        Санкт-Петербург \\
        2024
    \end{center}
\end{titlepage}

\setcounter{page}{2}
\tableofcontents

\newpage
\section*{Введение}
В настоящем реферате рассматриваются ключевые аспекты адаптивного управления, представленные в разделе 6.4. Основное внимание уделено параметризации модели объекта управления, методам непосредственной компенсации, расширенной ошибки, алгоритмам адаптации высокого порядка и итеративным процедурам синтеза управления. Эти методы используются для построения устойчивых систем адаптивного управления в условиях неопределённости параметров объекта.

\section{Параметризация модели объекта управления (6.4.2)}

Основная цель параметризации модели объекта заключается в представлении параметрической неопределённости в виде линейных комбинаций неизвестных параметров. Для этого используются вспомогательные фильтры, формирующие параметры объекта управления. 

\subsection{Лемма 6.5: Первая схема вспомогательных фильтров}
Первая схема вспомогательных фильтров основывается на уравнениях:
\[
v_1 = A v_1 + e_{n-1} u, \quad v_2 = A v_2 + e_{n-1} y,
\]
где \( v_1 \) и \( v_2 \) — векторы фильтрованных сигналов, \( A \) — матрица фильтра, \( e_{n-1} \) — вектор коэффициентов. Модель объекта может быть параметризована в виде:
\[
y = \phi^\top \theta + e(t),
\]
где \( \phi \) — регрессор, \( \theta \) — вектор неизвестных параметров. Функция \( e(t) \) экспоненциально затухает.

Данная схема обеспечивает компактность модели и удобство её применения в задачах адаптивного управления.

\subsection{Лемма 6.6: Вторая схема вспомогательных фильтров}
Вторая схема вспомогательных фильтров использует уравнения:
\[
v = A v + e_n u, \quad y = \phi^\top \theta + e_2,
\]
где \( e_2 \) также экспоненциально затухает. Эта схема приводит к более сложной, но точной модели объекта управления, что делает её полезной для задач управления системами с высокой степенью неопределённости.

Различие между схемами заключается в уровне неопределённости модели и сложности её применения для синтеза регулятора.

\section{Метод непосредственной компенсации (6.4.3)}

\subsection{Теорема 6.14 (\( p = 1 \))}
Для систем с \( p = 1 \) управление строится на основе метода непосредственной компенсации:
\[
u = \phi^\top \hat{\theta},
\]
где \( \hat{\theta} \) — вектор настраиваемых параметров, обновляемых по алгоритму адаптации:
\[
\dot{\hat{\theta}} = -\gamma \phi e,
\]
где \( \gamma > 0 \) — коэффициент адаптации. Ошибка управления \( e = y - y^* \) подчиняется уравнению:
\[
\dot{e} + \lambda e = 0,
\]
что обеспечивает экспоненциальное уменьшение ошибки слежения и устойчивость замкнутой системы. Этот метод эффективен для линейных объектов с низкой относительной степенью.

\section{Метод расширенной ошибки (6.4.3)}

\subsection{Теорема 6.15 (\( p > 1 \))}
Для систем с \( p > 1 \) вводится вектор расширенной ошибки:
\[
E = \begin{bmatrix}
e \\ \dot{e} \\ \vdots \\ e^{(p-1)}
\end{bmatrix},
\]
и формируется управление:
\[
u = -K E + \phi^\top \hat{\theta},
\]
где \( K \) — матрица обратной связи. Алгоритм адаптации параметров регулятора выглядит следующим образом:
\[
\dot{\hat{\theta}} = -\gamma \phi E.
\]
Замкнутая система устойчива, и все сигналы в системе ограничены. 

\subsection{Пример 6.3}
Пример демонстрирует применение метода расширенной ошибки к объекту управления второго порядка. Несмотря на повышенную сложность реализации, метод позволяет достичь высокой точности слежения и устойчивости замкнутой системы.

\section{Использование алгоритмов адаптации высокого порядка (6.4.4)}

\subsection{Теорема 6.16}
Алгоритмы адаптации высокого порядка позволяют улучшить переходные процессы за счёт использования производных ошибок. Управление задаётся уравнением:
\[
u = -K \phi + \phi^\top \hat{\theta},
\]
а настройка параметров осуществляется по алгоритму:
\[
\dot{\hat{\theta}} = -\gamma \phi e,
\]
где \( \gamma > 0 \) — коэффициент адаптации. Такой подход позволяет компенсировать нелинейности объекта управления и ускорить настройку системы.

\subsection{Пример 6.4}
Пример иллюстрирует использование алгоритма адаптации высокого порядка для системы с неизвестными параметрами. Результаты показывают, что метод обеспечивает быстрый выход системы на требуемую траекторию при высоком качестве переходных процессов.

\section{Итеративная процедура синтеза адаптивного управления (6.4.5)}

\subsection{Теорема 6.17}
Итеративные процедуры синтеза управления включают поэтапное добавление функций стабилизации, что описывается уравнением:
\[
\dot{e}_i = -\lambda_i e_i + f_i(e_{i-1}),
\]
где \( f_i \) — нелинейная функция стабилизации. Этот метод позволяет добиться устойчивости системы даже при высокой относительной степени.

\subsection{Пример 6.5 и таблица 6.1}
Пример 6.5 демонстрирует применение метода к системе с параметрической неопределённостью. Таблица 6.1 содержит сравнительный анализ различных подходов, включая их эффективность и устойчивость.

\section{Сравнение методов адаптивного управления (6.4.6)}
В этом разделе приводится анализ методов адаптивного управления. Метод непосредственной компенсации подходит для систем с низкой относительной степенью, в то время как методы с расширенной ошибкой и итеративные процедуры применимы для более сложных объектов.

\section*{Заключение}
Представленные методы обеспечивают эффективное управление линейными объектами с параметрической неопределённостью. Выбор подходящего метода зависит от характеристик системы и требований к её поведению.

\end{document}