\documentclass[a4paper,14pt]{extarticle} % Используем extarticle для поддержки шрифта 14
\usepackage[utf8]{inputenc}
\usepackage[T2A]{fontenc}
\usepackage[russian]{babel}
\usepackage{amsmath}
\usepackage{amssymb}
\usepackage{graphicx}
\usepackage[left=3cm,right=2cm,top=2cm,bottom=2cm]{geometry}
\linespread{1.5}

\begin{document}

\begin{titlepage}
    \begin{center}
        \large
        Министерство науки и высшего образования Российской Федерации \\
        Санкт-Петербургский государственный электротехнический университет \\
        \textbf{«ЛЭТИ» им. В.И. Ульянова (Ленина)} \\
        Кафедра систем автоматического управления

        \vfill

        \textbf{Реферат} \\
        по дисциплине \\
        \textbf{«Нелинейное адаптивное управление в технических системах»}

        \vfill

        Студент группы 9492 \hfill Викторов А.Д. \\
        Преподаватель \hfill Путов В.В.

        \vfill
        Санкт-Петербург \\
        2024
    \end{center}
\end{titlepage}

\setcounter{page}{2}
\tableofcontents

\newpage
\section{Введение}
Адаптивное управление — это метод управления динамическими системами с неизвестными или изменяющимися параметрами. Основная идея адаптивного управления заключается в том, чтобы подстраивать параметры регулятора в реальном времени, обеспечивая устойчивость и желаемые характеристики системы. В данном реферате мы рассмотрим несколько ключевых теорем и лемм, относящихся к адаптивной стабилизации и робастным алгоритмам высокого порядка.

\newpage
\section{Теорема 6.8: Адаптивная стабилизация по выходу строго пассивной системы}
Пусть выполняются допущения 6.3 и 6.4. Тогда закон управления (6.124), (6.125) обеспечивает для объекта (6.120), (6.121) при любых \(\gamma > 0\) и произвольных начальных условиях 
\(
x_0 = x(0), \hat{\theta}_0 = \hat{\theta}(0)
\) 
ограниченность \(x(t)\) и \(\hat{\theta}(0)\). Если регрессор w(t) ограничен, то, дополнительно, все сигналы в замкнутой системе ограничены и $\lim_{t\to\infty} S(x(t)) = 0.$
Основной вопрос, который возникает в связи с представленными результатами: насколько ограничительным является допущение 6.4? Как было показано в 2.6.3, условия (6.122), (6.123) выполняются только в том случае, если система

\begin{equation}
    \dot{x} = f(x) + g(x)u,
\end{equation}
\begin{equation}
    y = h(x)
\end{equation}

является строго пассивной. В частности, это означает, что относительная степень системы равна елиницы, а состояине равновесия x = 0 автономной части уравнения (1) является ассимптотически устойчивым. Таким образом, необходимо, необходиом признать, что допущение 6.4 является крайне ограничительным и как следствие - теорема 6.8 имеет ограничечнное практическое значение.

\subsection*{Понятие строгой пассивности}
Пусть у нас есть система, описываемая входом \( u(t) \) и выходом \( y(t) \). Система называется строго пассивной, если существует положительно определенная функция Ляпунова \( V(x) \) такая, что:
\[
\dot{V}(x) \leq u(t) y(t) - \alpha \|y(t)\|^2,
\]
где \( \alpha > 0 \) — некоторая положительная постоянная, а \( \|y(t)\| \) обозначает норму выходного сигнала.

\subsection*{Пример адаптивного регулятора}
Для стабилизации системы по выходу можно использовать закон адаптации для корректировки параметров регулятора в реальном времени:
\[
\dot{\theta} = \gamma y(t) u(t),
\]
где \( \theta \) — вектор адаптивных параметров, а \( \gamma \) — скорость адаптации.

\section{Теорема 6.9: Адаптивная стабилизация по выходу строго минимально-фазовой системы}
\textbf{Описание теоремы}: Теорема 6.9 касается минимально-фазовых систем, у которых все нули правой части передаточной функции находятся в левой полуплоскости. Такие системы можно стабилизировать только по выходу, используя адаптивный регулятор.

\subsection*{Минимальная фаза}
Система является минимально-фазовой, если её передаточная функция \( G(s) \) имеет все нули в левой полуплоскости. Это свойство выражается через условие:
\[
\text{Re}(s_i) < 0, \quad \forall i,
\]
где \( s_i \) — корни числителя передаточной функции.

\subsection*{Адаптивный регулятор}
Для минимально-фазовой системы можно использовать регулятор с обновляемыми параметрами:
\[
u(t) = -k y(t),
\]
где \( k \) — адаптивный параметр, который обновляется по правилу:
\[
\dot{k} = \gamma y(t) \left( y_d(t) - y(t) \right),
\]
где \( y_d(t) \) — желаемое значение выходного сигнала.

\section{Леммы 6.1 - 6.4 и Теорема 6.10: Адаптивное управление с расширенной ошибкой}
\textbf{Лемма 6.1}: В этой лемме рассматривается поведение ошибки адаптации при наличии расширенной ошибки. Для системы с выходом \( y(t) \) и референтной моделью \( y_r(t) \), расширенная ошибка определяется как:
\[
e(t) = y(t) - y_r(t).
\]

\textbf{Лемма 6.2}: Эта лемма доказывает, что функция Ляпунова \( V(x) \) для расширенной ошибки \( e(t) \) остаётся положительно определенной.

\textbf{Лемма 6.3}: Здесь рассматривается условие для сходимости параметров адаптации.

\textbf{Лемма 6.4}: Условие, обеспечивающее ограниченность функции Ляпунова.

\subsection*{Теорема 6.10}
В этой теореме описывается, как можно построить адаптивный регулятор с использованием расширенной ошибки. Пусть функция Ляпунова имеет вид:
\[
V(e, \theta) = \frac{1}{2} e^2 + \frac{1}{2} \| \theta - \theta^* \|^2,
\]
где \( \theta \) — вектор текущих параметров, а \( \theta^* \) — истинные параметры системы. Теорема утверждает, что можно построить такой закон адаптации, что \( e(t) \to 0 \) при \( t \to \infty \).

\section{Теорема 6.11: Алгоритм адаптации высокого порядка}
\textbf{Описание теоремы}: Теорема 6.11 утверждает, что можно разработать алгоритм адаптации высокого порядка, который учитывает производные ошибок более высокого порядка для более точной стабилизации системы.

\subsection*{Задача стабилизации}
Стабилизационный алгоритм учитывает не только текущую ошибку \( e(t) \), но и её производные:
\[
u(t) = -k_1 e(t) - k_2 \dot{e}(t) - \cdots - k_n e^{(n-1)}(t).
\]
\subsection*{Адаптивное обновление параметров}
Параметры \( k_i \) адаптируются на основе производных ошибок, чтобы обеспечить быструю сходимость и минимизировать ошибку управления.

\section{Теорема 13: Нелинейный робастный алгоритм высокого порядка}
\textbf{Описание теоремы}: Теорема 13 касается робастных алгоритмов адаптации, способных стабилизировать системы, даже если они подвержены нелинейным воздействиям и возмущениям.

\subsection*{Робастность}
Робастный алгоритм стабилизации учитывает внешние возмущения \( d(t) \) и использует нелинейные функции управления:
\[
u(t) = -k_1 \text{sat}(e(t)) - k_2 \text{sat}(\dot{e}(t)) - \cdots - k_n \text{sat}(e^{(n-1)}(t)),
\]
где \( \text{sat}(\cdot) \) — нелинейная функция насыщения, ограничивающая значения управления для повышения устойчивости системы к возмущениям.

\subsection*{Функция Ляпунова для робастности}
Используя ограничение на сигнал ошибки, можно записать:
\[
\dot{V}(x) \leq -\alpha V(x) + \beta \|d(t)\|^2,
\]
где \( \alpha \) и \( \beta \) — положительные постоянные. Это ограничение помогает гарантировать робастность системы.

\section*{Заключение}
Адаптивные и робастные алгоритмы управления позволяют стабилизировать сложные динамические системы даже при наличии неопределенности и возмущений. Применение адаптивных методов повышает устойчивость системы и её способность адаптироваться к изменениям параметров.

\end{document}