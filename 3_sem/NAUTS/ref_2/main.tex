\documentclass[14pt,a4paper]{article}
\usepackage[utf8]{inputenc}
\usepackage[T2A]{fontenc}
\usepackage[russian]{babel}
\usepackage{amsmath}
\usepackage{amssymb}
\usepackage{graphicx}
\usepackage[left=3cm,right=2cm,top=2cm,bottom=2cm]{geometry}
\linespread{1.5}

\begin{document}

\begin{titlepage}
    \begin{center}
        \large
        Министерство науки и высшего образования Российской Федерации \\
        Санкт-Петербургский государственный электротехнический университет \\
        \textbf{«ЛЭТИ» им. В.И. Ульянова (Ленина)} \\
        Кафедра систем автоматического управления

        \vfill

        \textbf{Реферат} \\
        по дисциплине \\
        \textbf{«Нелинейное адаптивное управление в технических системах»}

        \vfill

        Студент группы 9492 \hfill Викторов А.Д. \\
        Преподаватель \hfill Путов В.В.

        \vfill
        Санкт-Петербург \\
        2024
    \end{center}
\end{titlepage}

\tableofcontents

\section{Введение}
Метод скоростного градиента (СГ), разработанный А.Л. Фрадковым, является одним из ключевых инструментов в теории нелинейного и адаптивного управления. Этот метод позволяет синтезировать алгоритмы управления для широкого класса нелинейных систем, обеспечивая достижение различных целей управления, таких как стабилизация, слежение и оптимизация.

В данном реферате мы рассмотрим основные аспекты метода скоростного градиента, включая построение алгоритмов и условия достижения цели управления. Особое внимание будет уделено теоретическим основам метода, включая ключевые теоремы, обосновывающие его применимость и эффективность.

\section{Построение алгоритмов скоростного градиента}

\subsection{Основные понятия и определения}
Рассмотрим нелинейную систему вида:
\begin{equation}
    \dot{x} = f(x,u), \quad x \in R^n, \quad u \in R^m
\end{equation}
где $x$ - вектор состояния системы, $u$ - вектор управления, $f$ - нелинейная вектор-функция.

Целевой функционал управления определяется как:
\begin{equation}
    Q(x) \geq 0
\end{equation}

Задача управления состоит в минимизации $Q(x)$ путем выбора подходящего закона управления $u$.

\subsection{Алгоритм скоростного градиента для локального целевого функционала}
Алгоритм скоростного градиента в дифференциальной форме для локального целевого функционала имеет вид:
\begin{equation}
    \dot{u} = -\Gamma \nabla_u \dot{Q}(x,u)
\end{equation}
где $\Gamma$ - положительно определенная матрица коэффициентов усиления, а $\nabla_u \dot{Q}(x,u)$ - градиент скорости изменения целевого функционала по управлению.

Для вычисления $\nabla_u \dot{Q}(x,u)$ используется следующая формула:
\begin{equation}
    \nabla_u \dot{Q}(x,u) = \left(\frac{\partial f}{\partial u}\right)^T \nabla_x Q(x)
\end{equation}

\subsection{Алгоритм скоростного градиента для интегрального целевого функционала}
Для интегрального целевого функционала вида:
\begin{equation}
    J = \int_0^T Q(x(t))dt
\end{equation}
алгоритм скоростного градиента принимает форму:
\begin{equation}
    \dot{u} = -\Gamma \nabla_u Q(x)
\end{equation}

\subsection{Алгоритм скоростного градиента в конечной форме}
АСГ в конечной форме получается интегрированием дифференциальной формы:
\begin{equation}
    u(t) = u(0) - \int_0^t \Gamma \nabla_u \dot{Q}(x(\tau),u(\tau))d\tau
\end{equation}
или в дискретном времени:
\begin{equation}
    u(k+1) = u(k) - \Gamma \nabla_u \dot{Q}(x(k),u(k))
\end{equation}

\subsection{Комбинированные алгоритмы скоростного градиента}
Комбинированные АСГ объединяют дифференциальную и конечную формы:
\begin{equation}
    \dot{u} = -\Gamma_1 \nabla_u \dot{Q}(x,u) - \Gamma_2 \nabla_u Q(x)
\end{equation}
где $\Gamma_1$ и $\Gamma_2$ - положительно определенные матрицы коэффициентов усиления.

\section{Условия достижения цели управления}

\subsection{Теорема 3.1 (Условие достижения цели для дифференциальной формы АСГ)}
\textbf{Теорема 3.1.} Пусть для системы (1) и целевого функционала (2) выполнены следующие условия:
\begin{enumerate}
    \item Функция $Q(x)$ ограничена снизу;
    \item Существует функция $V(x,u) \geq 0$ такая, что $\dot{V} \leq -\beta(\dot{Q})$ для некоторой функции $\beta(z)$, удовлетворяющей условию $\beta(z) > 0$ при $z \neq 0$;
    \item Траектории замкнутой системы ограничены.
\end{enumerate}
Тогда:
\begin{enumerate}
    \item $\lim_{t \to \infty} \dot{Q}(x(t)) = 0$;
    \item Если дополнительно $\dot{Q}(x) = 0 \Rightarrow Q(x) = 0$, то $\lim_{t \to \infty} Q(x(t)) = 0$.
\end{enumerate}

\textbf{Доказательство:}
Рассмотрим функцию Ляпунова $V(x,u)$. Из условия 2 следует, что $V(x,u)$ не возрастает вдоль траекторий системы. Учитывая ограниченность $V(x,u)$ снизу, получаем:

\begin{equation}
    \lim_{t \to \infty} V(x(t),u(t)) = V^* < \infty
\end{equation}

Интегрируя неравенство $\dot{V} \leq -\beta(\dot{Q})$, получаем:

\begin{equation}
    \int_0^\infty \beta(\dot{Q}(x(\tau)))d\tau \leq V(x(0),u(0)) - V^* < \infty
\end{equation}

Отсюда следует, что $\lim_{t \to \infty} \beta(\dot{Q}(x(t))) = 0$. Учитывая свойства функции $\beta(z)$, получаем утверждение 1 теоремы.

Если выполнено дополнительное условие $\dot{Q}(x) = 0 \Rightarrow Q(x) = 0$, то из утверждения 1 следует утверждение 2.

\subsection{Теорема 3.2 (Условие достижения цели для конечной формы АСГ)}
\textbf{Теорема 3.2.} Пусть для системы (1) и целевого функционала (2) выполнены следующие условия:
\begin{enumerate}
    \item Функция $Q(x)$ ограничена снизу;
    \item $\|\nabla_u \dot{Q}(x,u)\| \leq L(1 + \|u\|)$ для некоторого $L > 0$;
    \item Траектории замкнутой системы ограничены.
\end{enumerate}
Тогда:
\begin{enumerate}
    \item $\lim_{t \to \infty} \|\nabla_u \dot{Q}(x(t),u(t))\| = 0$;
    \item Если дополнительно $\nabla_u \dot{Q}(x,u) = 0 \Rightarrow \dot{Q}(x) = 0$, то $\lim_{t \to \infty} \dot{Q}(x(t)) = 0$;
    \item Если дополнительно $\dot{Q}(x) = 0 \Rightarrow Q(x) = 0$, то $\lim_{t \to \infty} Q(x(t)) = 0$.
\end{enumerate}

\textbf{Доказательство:}
Рассмотрим функцию Ляпунова $V(u) = \frac{1}{2}\|u\|^2$. Вычисляя производную $V(u)$ вдоль траекторий системы, получаем:

\begin{equation}
    \dot{V} = u^T \dot{u} = -u^T \Gamma \nabla_u \dot{Q}(x,u) \leq -\lambda_{\min}(\Gamma) \|\nabla_u \dot{Q}(x,u)\|^2
\end{equation}

где $\lambda_{\min}(\Gamma)$ - минимальное собственное значение матрицы $\Gamma$.

Интегрируя это неравенство, получаем:

\begin{equation}
    \int_0^\infty \|\nabla_u \dot{Q}(x(\tau),u(\tau))\|^2 d\tau \leq \frac{1}{2\lambda_{\min}(\Gamma)} \|u(0)\|^2 < \infty
\end{equation}

Отсюда следует утверждение 1 теоремы. Утверждения 2 и 3 следуют из дополнительных условий.

\subsection{Теорема 3.3 (Условие достижения цели для комбинированного АСГ)}
\textbf{Теорема 3.3.} Пусть для системы (1) и целевого функционала (2) выполнены условия теорем 3.1 и 3.2. Тогда для комбинированного АСГ справедливы все утверждения теорем 3.1 и 3.2.

\textbf{Доказательство:}
Доказательство этой теоремы основывается на комбинации доказательств теорем 3.1 и 3.2. Рассматривается функция Ляпунова вида $V(x,u) = V_1(x) + \frac{1}{2}\|u\|^2$, где $V_1(x)$ - функция Ляпунова из теоремы 3.1. Анализ производной $\dot{V}$ позволяет получить все утверждения теоремы.

\subsection{Теорема 3.4 (Условие достижения цели для АСГ с интегральным целевым функционалом)}
\textbf{Теорема 3.4.} Пусть для системы (1) и интегрального целевого функционала (5) выполнены следующие условия:
\begin{enumerate}
    \item Функция $Q(x)$ неотрицательна и непрерывна;
    \item $\|\nabla_u Q(x)\| \leq L(1 + \|u\|)$ для некоторого $L > 0$;
    \item Траектории замкнутой системы ограничены.
\end{enumerate}
Тогда:
\begin{enumerate}
    \item $\lim_{t \to \infty} \|\nabla_u Q(x(t))\| = 0$;
    \item Если дополнительно $\nabla_u Q(x) = 0 \Rightarrow Q(x) = 0$, то $\lim_{t \to \infty} Q(x(t)) = 0$.
\end{enumerate}

\textbf{Доказательство:}
Доказательство этой теоремы аналогично доказательству теоремы 3.2, с использованием функции Ляпунова $V(u) = \frac{1}{2}\|u\|^2$ и анализом её производной вдоль траекторий системы.

\section{Заключение}
В данном реферате были рассмотрены основные аспекты метода скоростного градиента, включая построение алгоритмов СГ для различных форм целевых функционалов и условия достижения цели управления. Были представлены и доказаны ключевые теоремы, обосновывающие эффективность метода СГ для широкого класса нелинейных систем.

Метод скоростного градиента представляет собой мощный инструмент синтеза алгоритмов управления, обладающий рядом преимуществ:
\begin{itemize}
    \item Универсальность применения к различным классам нелинейных систем;
    \item Возможность работы с различными типами целевых функционалов;
    \item Гарантированное достижение цели управления при выполнении определенных условий;
    \item Простота реализации и интерпретации алгоритмов управления.
\end{itemize}

Дальнейшие исследования в области метода скоростного градиента могут быть направлены на расширение класса систем, к которым применим метод, улучшение сходимости алгоритмов и разработку новых модификаций метода для решения специфических задач управления.
\end{document}