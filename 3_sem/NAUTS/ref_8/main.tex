\documentclass[a4paper,14pt]{extarticle} % Используем extarticle для поддержки шрифта 14
\usepackage[utf8]{inputenc}
\usepackage[T2A]{fontenc}
\usepackage[russian]{babel}
\usepackage{amsmath}
\usepackage{amssymb}
\usepackage{graphicx}
\usepackage[left=3cm,right=2cm,top=2cm,bottom=2cm]{geometry}
\linespread{1.5}

\begin{document}

\begin{titlepage}
    \begin{center}
        \large
        Министерство науки и высшего образования Российской Федерации \\
        Санкт-Петербургский государственный электротехнический университет \\
        \textbf{«ЛЭТИ» им. В.И. Ульянова (Ленина)} \\
        Кафедра систем автоматического управления

        \vfill

        \textbf{Реферат} \\
        по дисциплине \\
        \textbf{«Нелинейное адаптивное управление в технических системах»}

        \vfill

        Студент группы 9492 \hfill Викторов А.Д. \\
        Преподаватель \hfill Путов В.В.

        \vfill
        Санкт-Петербург \\
        2024
    \end{center}
\end{titlepage}

\setcounter{page}{2}
\tableofcontents

\newpage
\section{Постановка задачи}

В данной работе рассматривается задача применения методов адаптивного, робастного и нелинейного управления для выхода линейных объектов с неопределённостями при наличии внешних возмущений. Пусть исследуемый объект описывается в виде линейной системы с параметрическими неопределённостями:

\begin{equation}
    \dot{x}(t) = A(\theta)x(t) + B(\theta)u(t) + Dd(t),
\end{equation}

где \( x(t) \in \mathbb{R}^n \) — вектор состояния, \( u(t) \in \mathbb{R}^m \) — вектор управления, \( d(t) \in \mathbb{R}^p \) — внешние возмущения, \( A(\theta) \) и \( B(\theta) \) — матрицы, зависящие от вектора неопределённых параметров \( \theta \in \mathbb{R}^q \), \( D \) — известная матрица, описывающая воздействие возмущений.

Задача заключается в синтезе закона управления \( u(t) \), который обеспечивает:

\begin{itemize}
    \item устойчивость замкнутой системы при всех допустимых значениях неопределённых параметров \( \theta \);
    \item удовлетворительное поведение выходной переменной \( y(t) \) в присутствии внешних возмущений \( d(t) \).
\end{itemize}

Выходная переменная определяется следующим образом:

\begin{equation}
    y(t) = Cx(t),
\end{equation}

где \( C \in \mathbb{R}^{l \times n} \) — матрица выходов.

\subsection*{Характеристика неопределённостей и возмущений}

Параметрические неопределённости в системе описываются следующим образом:

\begin{equation}
    \theta \in \Theta,
\end{equation}

где \( \Theta \) — компактное множество, определяющее допустимые значения параметров. Внешние возмущения \( d(t) \) предполагаются ограниченными:

\begin{equation}
    \|d(t)\| \leq d_{\max},
\end{equation}

где \( d_{\max} \) — известная верхняя граница величины возмущений.

\subsection*{Цель управления}

Основной целью является разработка робастного и адаптивного закона управления \( u(t) \), который обеспечивает выполнение следующих требований:

\begin{enumerate}
    \item \textbf{Адаптивность}: закон управления должен приспосабливаться к изменениям параметров системы в пределах множества \( \Theta \).
    \item \textbf{Робастность}: устойчивость системы должна сохраняться при любых допустимых возмущениях \( d(t) \) и неопределённостях параметров \( \theta \).
    \item \textbf{Уменьшение воздействия возмущений}: минимизация влияния внешних возмущений на поведение выходной переменной \( y(t) \).
\end{enumerate}

\subsection*{Формулировка задачи}

Необходимо найти такой закон управления в виде:

\begin{equation}
    u(t) = \mathcal{U}(x(t), t),
\end{equation}

где \( \mathcal{U} \) — некоторая функция, зависящая от состояния системы и, возможно, от времени, обеспечивающая выполнение требований к устойчивости и качеству управления.

\newpage
\section{Параметризованная модель объекта управления}

Для построения эффективных методов управления линейным объектом с неопределёнными параметрами необходимо ввести параметризованную модель, которая учитывает все возможные изменения в структуре системы. Параметризованная модель позволяет формализовать неопределённости и описывать объект управления в удобной форме для последующего анализа и синтеза законов управления.

Рассмотрим линейную систему, описываемую следующими уравнениями состояния:

\begin{equation}
    \dot{x}(t) = A(\theta)x(t) + B(\theta)u(t) + Dd(t),
\end{equation}

где \( x(t) \in \mathbb{R}^n \) — вектор состояния, \( u(t) \in \mathbb{R}^m \) — вектор управления, \( d(t) \in \mathbb{R}^p \) — вектор внешних возмущений. Матрицы \( A(\theta) \) и \( B(\theta) \) зависят от вектора неопределённых параметров \( \theta \in \mathbb{R}^q \).

\subsection*{Описание параметрической неопределённости}

Неопределённости в системе могут возникать по разным причинам, включая:

\begin{itemize}
    \item изменения физических параметров объекта (например, массы, инерции, сопротивления и т.д.);
    \item погрешности измерений или неполное знание параметров модели;
    \item влияние внешней среды, которое невозможно точно учесть в модели.
\end{itemize}

Пусть вектор параметров \( \theta \) принадлежит компактному множеству \( \Theta \subset \mathbb{R}^q \), определяющему все возможные значения неопределённых параметров. Тогда матрицы системы имеют следующий вид:

\begin{equation}
    A(\theta) = A_0 + \sum_{i=1}^q \theta_i A_i,
\end{equation}

\begin{equation}
    B(\theta) = B_0 + \sum_{i=1}^q \theta_i B_i,
\end{equation}

где \( A_0 \) и \( B_0 \) — номинальные матрицы системы, а \( A_i \) и \( B_i \) — известные матрицы, задающие структуру неопределённостей.

\subsection*{Особенности параметризованной модели}

Параметризованная модель системы позволяет:

\begin{enumerate}
    \item \textbf{Учитывать неопределённости}: модель включает в себя все допустимые изменения параметров, что позволяет проводить анализ устойчивости и синтезировать робастные регуляторы.
    \item \textbf{Обеспечивать адаптивность}: в случае использования адаптивных методов управления параметры \( \theta \) могут оцениваться в реальном времени, что позволяет системе адаптироваться к изменяющимся условиям.
\end{enumerate}

\subsection*{Анализ структуры матриц системы}

Структура матриц \( A(\theta) \) и \( B(\theta) \) является ключевым элементом при синтезе законов управления. Важно отметить, что параметры \( \theta \) могут влиять на поведение системы как линейно, так и нелинейно, что усложняет задачу обеспечения устойчивости и желаемого качества управления. Тем не менее, использование параметризованной модели позволяет систематически подходить к анализу влияния неопределённостей.

\subsection*{Пример параметризованной модели}

Для наглядности рассмотрим пример простейшей линейной системы с одной неопределённостью:

\begin{equation}
    \dot{x}(t) = \begin{pmatrix}
        0 & 1 \\
        -\theta_1 & -\theta_2
    \end{pmatrix} x(t) + \begin{pmatrix}
        0 \\
        1
    \end{pmatrix} u(t),
\end{equation}

где \( \theta_1 \) и \( \theta_2 \) — параметры, изменяющиеся в пределах заданного множества \( \Theta \). В этом случае задача синтеза управления осложняется необходимостью учёта всех возможных значений \( \theta_1 \) и \( \theta_2 \) для обеспечения устойчивости системы.

Таким образом, параметризованная модель объекта управления является основой для разработки робастных и адаптивных методов управления, которые могут эффективно справляться с неопределённостями и внешними возмущениями.


\newpage
\section{Робастное управление с использованием алгоритмов адаптации высокого порядка: теорема}

В этой главе рассматриваются методы синтеза робастного управления с использованием алгоритмов адаптации высокого порядка для линейных систем с параметрическими неопределённостями. Использование адаптивных алгоритмов позволяет существенно повысить устойчивость системы к внешним возмущениям и неопределённостям.

\subsection*{Постановка задачи управления}

Рассмотрим линейную систему, описываемую уравнением:

\begin{equation}
    \dot{x}(t) = A(\theta)x(t) + B(\theta)u(t) + Dd(t),
\end{equation}

где \( x(t) \in \mathbb{R}^n \) — вектор состояния, \( u(t) \in \mathbb{R}^m \) — вектор управления, \( d(t) \in \mathbb{R}^p \) — вектор внешних возмущений, а \( A(\theta) \) и \( B(\theta) \) зависят от вектора неопределённых параметров \( \theta \in \Theta \).

Целью является разработка такого закона управления \( u(t) \), который гарантирует устойчивость системы и минимизирует влияние возмущений при всех допустимых значениях параметров \( \theta \).

\subsection*{Основной принцип адаптивного робастного управления}

Алгоритмы адаптации высокого порядка обеспечивают корректировку параметров управления в реальном времени на основе измерений состояния системы. Они могут использовать методы, которые учитывают высокие производные состояния для более точной оценки неопределённостей.

Пусть \( u(t) \) определяется следующим образом:

\begin{equation}
    u(t) = -Kx(t) + \alpha(t),
\end{equation}

где \( K \) — матрица обратной связи состояния, а \( \alpha(t) \) — адаптивный компонент управления, предназначенный для компенсации параметрических неопределённостей и внешних возмущений.

\subsection*{Теорема об устойчивости робастного управления}

\textbf{Теорема}. Пусть система описывается уравнением (1), и пусть адаптивный закон управления имеет вид:

\begin{equation}
    \dot{\alpha}(t) = -\gamma \text{sign}\left( \frac{\partial V}{\partial x} B(\theta)x(t) \right),
\end{equation}

где \( \gamma > 0 \) — коэффициент адаптации, \( V(x) \) — положительно определённая функция Ляпунова. Тогда при условии правильного выбора матрицы \( K \) и параметра \( \gamma \) замкнутая система устойчива в смысле Ляпунова и удовлетворяет следующим свойствам:

\begin{enumerate}
    \item \textbf{Асимптотическая устойчивость}: при отсутствии возмущений \( d(t) \) и точной оценке параметров \( \theta \), состояние системы \( x(t) \) стремится к нулю при \( t \to \infty \).
    \item \textbf{Робастность}: при наличии внешних возмущений и неопределённостей параметры адаптивного алгоритма гарантируют ограниченность всех траекторий системы.
\end{enumerate}

\subsection*{Доказательство теоремы}

Для доказательства устойчивости рассмотрим функцию Ляпунова:

\begin{equation}
    V(x) = x^T P x,
\end{equation}

где \( P = P^T > 0 \) — симметричная положительно определённая матрица, удовлетворяющая уравнению Алгебраической Риккати:

\begin{equation}
    A^T P + P A = -Q,
\end{equation}

где \( Q = Q^T > 0 \) — заданная положительно определённая матрица. Вычислим производную функции Ляпунова:

\begin{equation}
    \dot{V}(x) = x^T (A^T P + P A) x + 2 x^T P B(\theta)u(t).
\end{equation}

Подставив выражение для \( u(t) \), можно показать, что \( \dot{V}(x) \leq -c \|x\|^2 \) для некоторого \( c > 0 \), что доказывает асимптотическую устойчивость системы.

Таким образом, алгоритмы адаптации высокого порядка в сочетании с правильно выбранной функцией Ляпунова позволяют обеспечить устойчивость и робастность системы в условиях неопределённостей и внешних возмущений.


\newpage
\section{Нелинейный робастный регулятор}

В этой главе рассматривается синтез нелинейного робастного регулятора для управления системой с параметрическими неопределённостями и внешними возмущениями. Нелинейные методы управления могут значительно улучшить устойчивость и качество управления по сравнению с линейными, особенно в случае сильных неопределённостей и нелинейного поведения объекта.

\subsection*{Постановка задачи управления}

Рассмотрим нелинейную систему с параметрическими неопределённостями, описываемую уравнением:

\begin{equation}
    \dot{x}(t) = f(x(t), \theta) + g(x(t), \theta)u(t) + Dd(t),
\end{equation}

где \( x(t) \in \mathbb{R}^n \) — вектор состояния, \( u(t) \in \mathbb{R}^m \) — вектор управления, \( d(t) \in \mathbb{R}^p \) — вектор внешних возмущений, а \( f(x, \theta) \) и \( g(x, \theta) \) — нелинейные векторные функции, зависящие от вектора неопределённых параметров \( \theta \in \Theta \).

Цель заключается в разработке нелинейного регулятора \( u(t) \), который обеспечивает устойчивость замкнутой системы и уменьшает влияние возмущений и параметрических неопределённостей.

\subsection*{Синтез нелинейного робастного регулятора}

Для синтеза робастного регулятора будем использовать метод обратной связи по состоянию, который учитывает нелинейности системы и неопределённости параметров. Закон управления \( u(t) \) имеет следующий вид:

\begin{equation}
    u(t) = \alpha(x(t)) + \beta(x(t))\eta(t),
\end{equation}

где \( \alpha(x(t)) \) — основная нелинейная часть управления, зависящая от состояния, \( \beta(x(t)) \) — управляющая функция, и \( \eta(t) \) — робастный адаптивный компонент, предназначенный для компенсации внешних возмущений и неопределённостей.

\subsection*{Выбор функций управления}

Рассмотрим структуру нелинейного регулятора более подробно:

\begin{itemize}
    \item \textbf{Функция \( \alpha(x(t)) \)}. Основная часть управления выбирается таким образом, чтобы стабилизировать номинальную систему при отсутствии возмущений и неопределённостей. Например, можно использовать обратное проектирование, метод Ляпунова или другие нелинейные методы стабилизации.
    
    \item \textbf{Функция \( \beta(x(t)) \)}. Эта функция задаёт усиление для робастного компонента управления и должна быть положительно определённой, чтобы обеспечить устойчивость системы.
    
    \item \textbf{Робастный компонент \( \eta(t) \)}. Этот компонент проектируется с использованием принципов робастного управления, чтобы компенсировать воздействие неопределённостей и возмущений. Примером может служить алгоритм скользящего режима, который обеспечивает инвариантность к внешним воздействиям:
    
    \begin{equation}
        \eta(t) = -\gamma \, \text{sign}(s(x(t))),
    \end{equation}
    
    где \( \gamma > 0 \) — коэффициент усиления, а \( s(x(t)) \) — скользящая поверхность, определяемая как функция состояния.
\end{itemize}

\subsection*{Анализ устойчивости}

Для анализа устойчивости замкнутой системы вводится функция Ляпунова \( V(x) \), которая является положительно определённой и непрерывно дифференцируемой:

\begin{equation}
    V(x) = x^T P x,
\end{equation}

где \( P = P^T > 0 \) — симметричная положительно определённая матрица. Производная функции Ляпунова по времени должна удовлетворять условию:

\begin{equation}
    \dot{V}(x) = \frac{\partial V}{\partial x} \dot{x} \leq -c \|x\|^2 + \|d(t)\|,
\end{equation}

где \( c > 0 \) — положительная константа. При правильном выборе параметров регулятора и робастного компонента \( \eta(t) \) можно доказать, что замкнутая система устойчива в смысле Ляпунова.

\subsection*{Теорема об устойчивости}

\textbf{Теорема}. Если функции \( \alpha(x(t)) \), \( \beta(x(t)) \) и \( \eta(t) \) выбраны таким образом, что производная функции Ляпунова \( \dot{V}(x) \) удовлетворяет вышеуказанному неравенству, то замкнутая система устойчива, а состояние \( x(t) \) стремится к нулю при \( t \to \infty \).

\subsection*{Заключение}

Нелинейный робастный регулятор позволяет эффективно справляться с неопределённостями и возмущениями, обеспечивая устойчивость и высокое качество управления даже в сложных условиях. Использование адаптивных и робастных методов в сочетании с нелинейным управлением обеспечивает инвариантность системы и её устойчивость при широком диапазоне неопределённостей.

\end{document}