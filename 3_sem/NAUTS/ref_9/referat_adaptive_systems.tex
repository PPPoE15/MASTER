\documentclass[a4paper,14pt]{extarticle} % Используем extarticle для поддержки шрифта 14
\usepackage[utf8]{inputenc}
\usepackage[T2A]{fontenc}
\usepackage[russian]{babel}
\usepackage{amsmath}
\usepackage{amssymb}
\usepackage{graphicx}
\usepackage[left=3cm,right=2cm,top=2cm,bottom=2cm]{geometry}
\linespread{1.5}

\begin{document}

\begin{titlepage}
    \begin{center}
        \large
        Министерство науки и высшего образования Российской Федерации \\
        Санкт-Петербургский государственный электротехнический университет \\
        \textbf{«ЛЭТИ» им. В.И. Ульянова (Ленина)} \\
        Кафедра систем автоматического управления

        \vfill

        \textbf{Реферат} \\
        по дисциплине \\
        \textbf{«Нелинейное адаптивное управление в технических системах»}

        \vfill

        Студент группы 9492 \hfill Викторов А.Д. \\
        Преподаватель \hfill Путов В.В.

        \vfill
        Санкт-Петербург \\
        2024
    \end{center}
\end{titlepage}

\setcounter{page}{2}
\tableofcontents

\newpage
\section*{Введение}
Адаптивные системы управления используются для обеспечения устойчивости и качества работы объектов управления при наличии неопределённостей и изменяющихся условий. В данном реферате рассматриваются подходы к синтезу адаптивных систем по выходу для строго минимально-фазовых одноканальных и многоканальных объектов, а также общий случай с использованием метода шунтирования.

\section{Синтез адаптивных систем по выходу для строго минимально-фазового одноканального объекта}
\subsection*{Описание задачи}
Минимально-фазовый одноканальный объект характеризуется тем, что его нули лежат в левой полуплоскости комплексной плоскости, а выходная переменная зависит только от входного воздействия и внутренних параметров. Требуется разработать адаптивную систему управления, которая минимизирует ошибку $e(t) = y_r(t) - y(t)$, где $y_r(t)$ — заданное значение выходной переменной.

\subsection*{Метод скоростного градиента}
Метод скоростного градиента основан на минимизации квадратичного критерия качества:
\begin{equation}
J(t) = \frac{1}{2}e^2(t).
\end{equation}
Процесс адаптации включает вычисление градиента функционала $J$ относительно параметров $\theta$:
\begin{equation}
\dot{\theta} = -\gamma \frac{\partial J}{\partial \theta},
\end{equation}
где $\gamma > 0$ — коэффициент скорости адаптации.

Производная функционала может быть записана в виде:
\begin{equation}
\frac{\partial J}{\partial \theta} = -e(t) \frac{\partial y(t)}{\partial \theta}.
\end{equation}
На основе этого уравнения формируется закон изменения параметров системы управления.

\subsection*{Алгоритм управления}
Управляющее воздействие формируется по закону:
\begin{equation}
u(t) = \theta^T \phi(t),
\end{equation}
где $\phi(t)$ — вектор входных сигналов, а $\theta$ — вектор адаптивных параметров. В результате система способна автоматически подстраиваться под изменения объекта управления.

\section{Синтез адаптивной системы по выходу для строго минимально-фазового многоканального объекта}
\subsection*{Особенности многоканального объекта}
Многоканальный объект управления описывается уравнениями вида:
\begin{equation}
\mathbf{Y}(s) = \mathbf{W}(s) \mathbf{U}(s),
\end{equation}
где $\mathbf{Y}(s)$ — вектор выходных переменных, $\mathbf{U}(s)$ — вектор управляющих воздействий, а $\mathbf{W}(s)$ — матричная передаточная функция.

Основной сложностью в многоканальных системах является взаимное влияние каналов, что требует учета всех входов и выходов при синтезе системы управления.

\subsection*{Методы синтеза}
Для синтеза адаптивной системы используется следующая процедура:
\begin{enumerate}
    \item Построение модели объекта в пространстве состояний:
    \begin{equation}
    \dot{\mathbf{x}}(t) = \mathbf{A}\mathbf{x}(t) + \mathbf{B}\mathbf{u}(t),
    \quad \mathbf{y}(t) = \mathbf{C}\mathbf{x}(t).
    \end{equation}
    \item Формирование закона управления с учетом адаптации параметров:
    \begin{equation}
    \mathbf{u}(t) = \mathbf{\Theta}(t)\mathbf{\Phi}(t),
    \end{equation}
    где $\mathbf{\Theta}(t)$ — матрица адаптивных параметров, $\mathbf{\Phi}(t)$ — вектор входных сигналов.
    \item Построение критерия качества, обеспечивающего устойчивость системы:
    \begin{equation}
    J = \frac{1}{2} \|\mathbf{e}(t)\|^2,
    \end{equation}
    где $\mathbf{e}(t) = \mathbf{y}_r(t) - \mathbf{y}(t)$ — вектор ошибки.
\end{enumerate}

\section{Общий случай. Метод шунтирования}
\subsection*{Описание метода шунтирования}
Метод шунтирования применяется для объектов с высокой степенью неопределенности, а также в случаях, когда параметры объекта изменяются со временем. Основная идея метода заключается в добавлении вспомогательных контуров управления, компенсирующих влияние неопределенностей.

\subsection*{Уравнения системы}
Объект управления описывается как:
\begin{equation}
\dot{\mathbf{x}}(t) = \mathbf{A}\mathbf{x}(t) + \mathbf{B}\mathbf{u}(t) + \mathbf{\Delta f}(t),
\end{equation}
где $\mathbf{\Delta f}(t)$ — неопределенности в модели.

Для компенсации неопределенностей вводится корректирующий сигнал $\mathbf{v}(t)$:
\begin{equation}
\mathbf{u}(t) = \mathbf{u}_0(t) + \mathbf{v}(t),
\end{equation}
где $\mathbf{u}_0(t)$ — базовый закон управления.

\subsection*{Алгоритм управления}
Алгоритм управления строится следующим образом:
\begin{enumerate}
    \item Вычисляется базовый сигнал управления $\mathbf{u}_0(t)$ на основе номинальной модели.
    \item Определяется корректирующий сигнал $\mathbf{v}(t)$ с использованием наблюдателей или адаптивных методов.
    \item Итоговое управление формируется как сумма базового и корректирующего сигналов.
\end{enumerate}

\subsection*{Преимущества метода}
Метод шунтирования обеспечивает:
\begin{itemize}
    \item Устойчивость системы при изменении параметров объекта.
    \item Высокое качество управления в условиях неопределенности.
    \item Простоту настройки адаптивных параметров.
\end{itemize}


\end{document}
