\documentclass[a4paper,14pt]{extarticle} % Используем extarticle для поддержки шрифта 14
\usepackage[utf8]{inputenc}
\usepackage[T2A]{fontenc}
\usepackage[russian]{babel}
\usepackage{amsmath}
\usepackage{amssymb}
\usepackage{graphicx}
\usepackage[left=3cm,right=2cm,top=2cm,bottom=2cm]{geometry}
\linespread{1.5}

\begin{document}

\begin{titlepage}
    \begin{center}
        \large
        Министерство науки и высшего образования Российской Федерации \\
        Санкт-Петербургский государственный электротехнический университет \\
        \textbf{«ЛЭТИ» им. В.И. Ульянова (Ленина)} \\
        Кафедра систем автоматического управления

        \vfill

        \textbf{Реферат} \\
        по дисциплине \\
        \textbf{«Нелинейное адаптивное управление в технических системах»}

        \vfill

        Студент группы 9492 \hfill Викторов А.Д. \\
        Преподаватель \hfill Путов В.В.

        \vfill
        Санкт-Петербург \\
        2024
    \end{center}
\end{titlepage}

\setcounter{page}{2}
\tableofcontents

\newpage

\section{Введение}

В современной теории управления одной из наиболее актуальных проблем является разработка эффективных методов управления сложными нелинейными системами в условиях неопределенности. Эта задача приобретает особую важность в связи с растущей сложностью технических систем и увеличением требований к их производительности и надежности. В данном реферате рассматриваются передовые методы управления нелинейными системами, включая адаптивное, адаптивное робастное и нелинейное управление, с особым акцентом на методы непосредственной (адаптивной) компенсации.

Адаптивное управление представляет собой подход, при котором система управления способна изменять свои параметры или структуру в ответ на изменения в динамике объекта управления или внешней среды. Это позволяет системе поддерживать оптимальную производительность даже в условиях значительной неопределенности.

Адаптивное робастное управление объединяет преимущества адаптивного и робастного подходов, обеспечивая способность системы адаптироваться к изменениям, сохраняя при этом устойчивость в широком диапазоне условий работы.

Нелинейное управление, в свою очередь, предоставляет инструменты для работы с системами, чье поведение не может быть адекватно описано линейными моделями. Это особенно важно для многих реальных систем, которые проявляют существенно нелинейное поведение.

В данном реферате мы рассмотрим методы непосредственной (адаптивной) компенсации, которые позволяют эффективно справляться с неопределенностями в системе путем их прямой оценки и компенсации. Особое внимание будет уделено методам адаптивной и адаптивной робастной стабилизации, а также новому классу робастно-адаптивных законов стабилизации.

\section{Постановка задачи управления неопределенными объектами}

Задача управления неопределенными объектами является одной из ключевых проблем в теории автоматического управления. Она возникает в ситуациях, когда параметры объекта управления не известны точно или могут изменяться во времени. В общем виде эту задачу можно сформулировать следующим образом:

Рассмотрим нелинейную систему, описываемую уравнением:

\begin{equation}
\dot{x} = f(x, u, \theta, t)
\end{equation}

где:
\begin{itemize}
    \item $x \in \mathbb{R}^n$ - вектор состояния системы,
    \item $u \in \mathbb{R}^m$ - вектор управляющих воздействий,
    \item $\theta \in \mathbb{R}^p$ - вектор неизвестных параметров системы,
    \item $t$ - время,
    \item $f(\cdot)$ - нелинейная функция, описывающая динамику системы.
\end{itemize}

Задача управления заключается в разработке закона управления $u = u(x, t)$, который обеспечивает достижение заданных целей управления (например, стабилизация системы, слежение за заданной траекторией) при наличии неопределенностей в параметрах $\theta$ и, возможно, в структуре функции $f(\cdot)$.

Основные сложности при решении этой задачи связаны с:

\begin{enumerate}
    \item Нелинейностью системы, что затрудняет применение классических методов линейной теории управления.
    \item Наличием неопределенностей в параметрах системы, что требует разработки адаптивных механизмов.
    \item Возможным наличием внешних возмущений и шумов измерений, что требует обеспечения робастности системы управления.
\end{enumerate}

В контексте данного реферата мы сосредоточимся на методах управления по состоянию, предполагая, что вектор состояния $x$ доступен для измерения. Это предположение позволяет разрабатывать более эффективные законы управления, но на практике может потребовать использования наблюдателей состояния для оценки недоступных для прямого измерения компонент вектора $x$.

\section{Управление по состоянию}

Управление по состоянию является одним из фундаментальных подходов в теории управления, который предполагает, что закон управления формируется на основе полной информации о текущем состоянии системы. В контексте нелинейных систем с неопределенностями, управление по состоянию предоставляет ряд преимуществ:

\begin{enumerate}
    \item Возможность прямого влияния на динамику системы: имея доступ ко всем компонентам вектора состояния, можно более точно корректировать поведение системы.
    \item Потенциально более высокое быстродействие: в отличие от управления по выходу, нет необходимости в оценке недоступных переменных состояния, что может снизить задержки в контуре управления.
    \item Упрощение синтеза законов управления: наличие полной информации о состоянии системы позволяет применять более широкий класс методов синтеза, включая методы, основанные на функциях Ляпунова.
\end{enumerate}

Рассмотрим общую форму закона управления по состоянию для нелинейной системы:

\begin{equation}
u = k(x, t, \hat{\theta})
\end{equation}

где $k(\cdot)$ - нелинейная функция, зависящая от текущего состояния системы $x$, времени $t$, и оценки неизвестных параметров $\hat{\theta}$.

Основная задача при синтезе такого закона управления заключается в выборе функции $k(\cdot)$ таким образом, чтобы обеспечить желаемые свойства замкнутой системы (устойчивость, качество переходных процессов, робастность к неопределенностям) при различных значениях неизвестных параметров $\theta$.

\section{Методы непосредственной (адаптивной) компенсации}

Методы непосредственной (адаптивной) компенсации представляют собой класс подходов к управлению нелинейными системами с неопределенностями, основанных на прямой оценке и компенсации неизвестных параметров или нелинейностей системы. Эти методы позволяют эффективно справляться с неопределенностями, обеспечивая высокое качество управления.

\subsection{Метод адаптивной стабилизации}

Рассмотрим нелинейную систему вида:

\begin{equation}
\dot{x} = f(x) + g(x)u + \theta^T\phi(x)
\end{equation}

где $x \in \mathbb{R}^n$ - вектор состояния, $u \in \mathbb{R}$ - управляющее воздействие, $f(x)$ и $g(x)$ - известные нелинейные функции, $\theta \in \mathbb{R}^p$ - вектор неизвестных параметров, $\phi(x) \in \mathbb{R}^p$ - известная векторная функция.

Метод адаптивной стабилизации основан на использовании закона управления вида:

\begin{equation}
u = -k(x) - \hat{\theta}^T\phi(x)
\end{equation}

где $k(x)$ - стабилизирующая функция обратной связи, а $\hat{\theta}$ - оценка неизвестного вектора параметров $\theta$.

Закон адаптации параметров может быть представлен в виде:

\begin{equation}
\dot{\hat{\theta}} = \Gamma\phi(x)x^TPB
\end{equation}

где $\Gamma > 0$ - матрица коэффициентов усиления адаптации, $P$ - положительно определенная матрица, удовлетворяющая уравнению Ляпунова $A^TP + PA = -Q$, $A$ и $B$ - матрицы линеаризованной системы.

\subsection{Метод адаптивной робастной стабилизации}

Для повышения робастности адаптивной системы к неопределенностям и внешним возмущениям используется метод адаптивной робастной стабилизации. Рассмотрим систему:

\begin{equation}
\dot{x} = f(x) + g(x)u + \theta^T\phi(x) + d(x,t)
\end{equation}

где $d(x,t)$ представляет ограниченное внешнее возмущение.

Закон управления в этом случае может быть модифицирован следующим образом:

\begin{equation}
u = -k(x) - \hat{\theta}^T\phi(x) - \rho(x)\text{sign}(s(x))
\end{equation}

где $s(x)$ - поверхность скольжения, а $\rho(x)$ - функция, обеспечивающая робастность к ограниченным возмущениям.

Закон адаптации параметров модифицируется для обеспечения робастности:

\begin{equation}
\dot{\hat{\theta}} = \Gamma\phi(x)s(x) - \sigma\Gamma\hat{\theta}
\end{equation}

где $\sigma > 0$ - параметр, обеспечивающий ограниченность оценок параметров.

\section{Метод нелинейной робастной стабилизации}

\subsection{Типы неопределенностей класса нелинейных объектов}

При рассмотрении нелинейных систем можно выделить следующие типы неопределенностей:

\begin{enumerate}
    \item Параметрические неопределенности: неизвестные или изменяющиеся параметры системы.
    \item Структурные неопределенности: неточности в модели системы.
    \item Внешние возмущения: неизвестные входные сигналы, действующие на систему.
\end{enumerate}

Рассмотрим нелинейную систему с неопределенностями:

\begin{equation}
\dot{x} = f(x) + g(x)u + \Delta f(x) + \Delta g(x)u + d(t)
\end{equation}

где $\Delta f(x)$ и $\Delta g(x)$ представляют структурные неопределенности, а $d(t)$ - внешнее возмущение.

\subsection{Унифицированный метод нелинейной робастной стабилизации}

Унифицированный метод нелинейной робастной стабилизации основан на комбинации адаптивного управления и методов скользящего режима. Рассмотрим закон управления вида:

\begin{equation}
u = u_n(x) + u_a(x) + u_s(x)
\end{equation}

где $u_n(x)$ - номинальная составляющая управления, $u_a(x)$ - адаптивная составляющая, $u_s(x)$ - скользящая составляющая.

Номинальная составляющая определяется как:

\begin{equation}
u_n(x) = -g^+(x)(f(x) + k(x))
\end{equation}

где $g^+(x)$ - псевдообратная матрица к $g(x)$, а $k(x)$ - стабилизирующая функция.

Адаптивная составляющая:

\begin{equation}
u_a(x) = -\hat{\theta}^T\phi(x)
\end{equation}

с законом адаптации:

\begin{equation}
\dot{\hat{\theta}} = \Gamma\phi(x)s^T(x)
\end{equation}

Скользящая составляющая:

\begin{equation}
u_s(x) = -\rho(x)\text{sign}(s(x))
\end{equation}

где $s(x)$ - поверхность скольжения, а $\rho(x)$ выбирается для обеспечения робастности.

\section{Новый класс робастно-адаптивных законов стабилизации}

Новый класс робастно-адаптивных законов стабилизации объединяет преимущества адаптивного и робастного подходов, обеспечивая высокую эффективность управления в условиях различных типов неопределенностей.

Рассмотрим закон управления вида:

\begin{equation}
u = -k(x) - \hat{\theta}^T\phi(x) - \beta(x)\tanh(\alpha s(x))
\end{equation}

где $\beta(x)$ - функция, обеспечивающая робастность, $\alpha > 0$ - параметр, определяющий скорость перехода к скользящему режиму, $\tanh(\cdot)$ - гиперболический тангенс, используемый для сглаживания разрывной функции sign.

Закон адаптации параметров:

\begin{equation}
\dot{\hat{\theta}} = \Gamma\phi(x)s^T(x) - \sigma\Gamma\hat{\theta} - \Gamma\phi(x)\beta(x)\tanh(\alpha s(x))
\end{equation}

Этот подход обеспечивает:

\begin{enumerate}
    \item Адаптацию к параметрическим неопределенностям.
    \item Робастность к структурным неопределенностям и внешним возмущениям.
    \item Сглаженное управляющее воздействие, снижающее эффект чаттеринга.
    \item Гарантированную ограниченность оценок параметров.
\end{enumerate}

\section{Заключение}

В данном реферате были рассмотрены современные методы адаптивного, адаптивного робастного и нелинейного управления нелинейными системами по состоянию, с особым акцентом на методы непосредственной (адаптивной) компенсации. Были представлены основные подходы к решению задачи управления неопределенными объектами, включая методы адаптивной и адаптивной робастной стабилизации, метод нелинейной робастной стабилизации и новый класс робастно-адаптивных законов стабилизации.

Каждый из рассмотренных методов имеет свои преимущества и ограничения, и выбор конкретного подхода зависит от специфики решаемой задачи, характера неопределенностей в системе и требований к качеству управления. Объединение различных подходов, как показано в новом классе робастно-адаптивных законов стабилизации, позволяет создавать более эффективные и универсальные системы управления, способные справляться с широким спектром неопределенностей и возмущений.

Дальнейшие исследования в этой области могут быть направлены на разработку методов, сочетающих преимущества рассмотренных подходов с современными методами машинного обучения и искусственного интеллекта, что потенциально может привести к созданию еще более эффективных и адаптивных систем управления нелинейными объектами.

\end{document}
