\documentclass[a4paper,14pt]{extarticle} % Используем extarticle для поддержки шрифта 14
\usepackage[utf8]{inputenc}
\usepackage[T2A]{fontenc}
\usepackage[russian]{babel}
\usepackage{amsmath}
\usepackage{amssymb}
\usepackage{graphicx}
\usepackage[left=3cm,right=2cm,top=2cm,bottom=2cm]{geometry}
\linespread{1.5}

\begin{document}

\begin{titlepage}
    \begin{center}
        \large
        Министерство науки и высшего образования Российской Федерации \\
        Санкт-Петербургский государственный электротехнический университет \\
        \textbf{«ЛЭТИ» им. В.И. Ульянова (Ленина)} \\
        Кафедра систем автоматического управления

        \vfill

        \textbf{Реферат} \\
        по дисциплине \\
        \textbf{«Нелинейное адаптивное управление в технических системах»}

        \vfill

        Студент группы 9492 \hfill Викторов А.Д. \\
        Преподаватель \hfill Путов В.В.

        \vfill
        Санкт-Петербург \\
        2024
    \end{center}
\end{titlepage}

\setcounter{page}{2}
\tableofcontents

\section{Введение}
Пусть задача адаптивного управления поставлена на содержалельном уровне и формализована. Это означает, что задано математическое описание объекта управления и внешних воздействий с точностью до неизвестных параметров $\xi$.
Указано также множество $\Xi$ значений этих параметров, дана,
спецификация управляющих воздействий и измеряемых выходов объекта. Кроме того, должна, быть сформулирована цель управления.

Процесс синтеза адаптивного регулятора можно разбить на следующие этапы:

\subsection{Этап 1} Выбор идеального” закона управления. Находится закон управления, обеспечивающий принципиальную возможность достижения указанной цели управления.
Вектор параметров $\xi$ предполагается известным. Полученный закон управления непосредственно реализован быть не может, так как он зависит, в общем случае, от неизвестных параметров объекта. В этом смысле его можно назвать идеальным законом управления. Например, такой закон управления может строиться на основе решения задачи оптимального управления. Но и не оптимальные (в общепринятом смысле этого слова) законы управления также могут рассматриваться как ”идеальные”, поскольку речь идет о том, что при их синтезе предполагается наличие достаточно точной информации о параметрах объекта и среды.

Обычно при синтезе идеального закона управления делают некоторые упрощающие предположения относительно динамики объекта, а также пренебрегают некоторыми возмущениями и помехами измерений.

Иногда основную цель управления заменяют некоторой вспомогательной (вторичной) целью, выполнение которой косвенно позволяет достигнуть и исходную цель.

\subsection{Этап 2}
Выбор настраиваемых параметров и цели адаптации. Неизвестные параметры, от которых зависит найденный идеальный закон управления заменяются настраиваемыми параметрами. В результате получается алгоритм управления, в который уже не входят неизвестные параметры, поэтому он может быть реализован регулятором. Известны два подхода к синтезу адаптивных регуляторов.

При прямом подходе настраиваемыми параметрами являются непосредственно коэффициенты закона управления (т.е. регулятора нижнего уровня). Количество настраиваемых параметров выбирается по возможности наименьшим.
При идентификационном (непрямом) подходе выполняется оценивание значений, необходимых для синтеза, регулятора неизвестных параметров объекта и характеристик внешних
воздействий. Далее выполняется процедура совмещенного синтеза — оценки параметров используются для вычисления коэффициентов, входящих в закон управления.

Когда, настраиваемые параметры выбраны, ставится цель
адаптации. Это - некоторое вспомогательное целевое условие, являющееся основой для последующей разработки алгоритма адаптации. При прямом подходе пель адаптации
совпадает с исходной, либо вспомогательной, целью управления. При идентификационном подходе цель адаптации обычно сводится к обеспечению совпадения, или близости,
оценок неизвестных параметров к их истинным” значениям. Вспомогательная цель адаптации при таком подходе может выражаться, например, как совпадение реакций объекта управления и настраиваемой модели объекта на внешнее воздействие. Настраиваемая модель описывается уравнени ями, аналогичными уравнениям объекта управления, в которых неизвестные параметры заменены их (настраиваемыми) оценками.

Требуемые свойства системы управления обычно задаются эталонной моделью. Эта модель может включаться в систему явно, в виде некоторого динамического.
звена, обладающего заданной реакцией на командное (задающее) воздействие, либо неявно — присутствовать в виде некоторых ”уставок” (параметров) алгоритма адаптации. Соответственно, системы первого типа называются системами с явной эталонной моделью, а системы второго типа - с неявной эталонной моделью.

Системы с явной эталонной моделью могут быть подразделены, в свою очередь, исходя из способа, достижения цели на системы с параметрической и сигнальной адаптацией."

В системах с сигнальной настройкой эффект адаптации достигается без изменения параметров регулятора путем увеличения его коэффициентов или обеспечением скользящих
режимов. Такие системы безусловно проще в реализации, однако они обеспечивают желаемое поведение

только в относительно узком диапазоне значений параметров объекта.
В системах с параметрической адаптацией цель достигается изменением параметров регулятора, Эти системы более универсальны, однако обладают более сложной структурой.
Алгоритмы адаптации используют сигнал рассогласования между выходами системы и эталонной модели, Сложность этих систем определяется количеством настраиваемых параметров.

Для повышения точности систем и скорости адаптации можно использовать сигнально-параметрические алгоритмы, в которых сочетается сигнальная и параметрическая адаптация. В таких системах сигнальная адаптация обеспечивается обычно быстрым релейным алгоритмом. Параметрическая адаптация имеет узкую полосу пропускания” и служит для
стабилизации коэффициентов передачи в заданных пределах,
Такие системы, кроме быстродействия и точности, также более просты в реализации, поскольку присутствие сигнальной компоненты позволяет уменьшить число настраиваемых параметров.

\subsection{Этап 3} Выбор алгоритма адаптации. Как правило, алгоритмы адаптации представляют собой рекуррентные процедуры, относящиеся к классу методов последовательного улучшения. Так как в условиях неопределекности добиться сразу выполнения цели управления, вообще говоря, невозможно, то алгоритм адаптации осуществляет последовательное изменение настраиваемых параметров, приближаясь к выполнению цели. Такого рода алгоритмы обычно строятся на основе процедур градиентного типа.

Решающее влияние на работоспособность алгоритма адаптации оказывает выбор коэффициента усиления (параметра шага) алгоритма. Для решения этой задачи известны такие методы, как метод наименьших квадратов, метод стотастической аппроксимации и метод рекуррентных целевых неровенств.

\subsection{Этап 4}
Исследование работоспособности адаптивной системы. Заключительным этапом синтеза адаптивного регулятора, предваряющим разработку его технической реализации, является исследование работоспособности системы с учетом характера возмущений, внешних воздействий, ограничений на переменные состояния объекта и других факторов, которые не учитывались при синтезе. На этом этапе также уточняются параметры алгоритма адаптации и, возможно, выполняется его модификация.

Значительную роль в обосновании работоспособности адапливных систем управления играет прямой метод Ляпунова. Но этот метод является в основном инструментом для теоретических исследований и не может дать ответы на все вопросы, касающиеся устойчивости и качества работы адаптивных регуляторов в реальных условиях. Поэтому большое место в исследовании адаптивных систем управления играет моделирование. Особенно велико значение моделирования на этапе получения количественных характеристик системы. Для упрощения процедуры моделирования и многовариантного анализа систем применяются проблемно-ориентированные пакеты прикладных программ.

Надо заметить, что характерной особенностью процесса проектирования адаптивных систем управления является его цикличность. Как правило, алгоритм адаптации удается синтезировать при значительном упрощении модели объекта, и на следующих стадиях проектирования может оказаться, что выбранный алгоритм, или даже метод адаптивного управления, не отвечает условиям поставленной задачи и процесс проектирования повторяется.

Обратимся к задаче адаптивного управления непрерывными объектами. Основное число беспоисковых алгоритмов адаптации, для которых имеются условия работоспособности, являются алгоритмами скоростного градиента.

\newpage
\section{Адаптивные системы с эталонной моделью}
Рассматривается обобщенный настраиваемый объект (ОНО)
\begin{equation}
\dot{x}(t) = (A + \Delta A)x(t) + (B + \Delta B)r(t)
\end{equation}

\subsection{Общая структура систем с явной эталонной моделью}
Адаптивные системы с явной эталонной моделью характеризуются наличием отдельно выделенной модели желаемого поведения системы. Основная идея заключается в минимизации ошибки между выходом реальной системы и эталонной модели.

Рассмотрим общую структуру такой системы:

\begin{itemize}
    \item Объект управления: $\dot{x} = f(x, u, \theta)$, $y = h(x)$
    \item Эталонная модель: $\dot{x}_m = f_m(x_m, r)$, $y_m = h_m(x_m)$
    \item Ошибка: $e = y - y_m$
    \item Регулятор: $u = u(x, r, \theta)$
    \item Адаптивный механизм: $\dot{\theta} = F(e, x, r, \theta)$
\end{itemize}

где $x$ - вектор состояния объекта, $u$ - управление, $\theta$ - вектор настраиваемых параметров, $y$ - выход объекта, $x_m$ - состояние эталонной модели, $r$ - задающее воздействие, $y_m$ - выход эталонной модели.

\subsection{Алгоритмы параметрической адаптации}
При параметрической адаптации происходит настройка параметров регулятора. Рассмотрим основной алгоритм:

\begin{equation}
\dot{\theta} = -\gamma e \frac{\partial e}{\partial \theta}
\end{equation}

где $\theta$ - вектор настраиваемых параметров, $\gamma$ - коэффициент усиления адаптации, $e$ - ошибка между выходом системы и эталонной модели.

\subsubsection{Пример параметрической адаптации}
Рассмотрим систему первого порядка:

\begin{equation}
\dot{x} = ax + bu
\end{equation}

с эталонной моделью:

\begin{equation}
\dot{x}_m = -x_m + r
\end{equation}

Закон управления:

\begin{equation}
u = k_1 x + k_2 r
\end{equation}

Алгоритм адаптации:

\begin{equation}
\begin{aligned}
\dot{k}_1 &= -\gamma_1 e x \\
\dot{k}_2 &= -\gamma_2 e r
\end{aligned}
\end{equation}

\subsection{Алгоритмы сигнальной адаптации}
Сигнальная адаптация предполагает введение дополнительного управляющего воздействия для компенсации ошибки. Основное уравнение:

\begin{equation}
u_{a} = -k e
\end{equation}

где $u_{a}$ - адаптивное управление, $k$ - коэффициент усиления, $e$ - ошибка.

\subsubsection{Пример сигнальной адаптации}
Рассмотрим ту же систему первого порядка:

\begin{equation}
\dot{x} = ax + bu
\end{equation}

Закон управления:

\begin{equation}
u = u_0 + u_a
\end{equation}

где $u_0$ - номинальное управление, $u_a$ - адаптивная добавка.

Алгоритм адаптации:

\begin{equation}
u_a = -k \int_0^t e(\tau) d\tau
\end{equation}

\subsection{Алгоритмы сигнально-параметрической адаптации}
Сигнально-параметрическая адаптация объединяет преимущества обоих подходов. Управляющее воздействие формируется как:

\begin{equation}
u = u_{0}(\theta) + u_{a}
\end{equation}

где $u_{0}(\theta)$ - параметрически настраиваемая часть, $u_{a}$ - сигнальная составляющая.

\subsubsection{Пример сигнально-параметрической адаптации}
Для системы первого порядка:

\begin{equation}
\dot{x} = ax + bu
\end{equation}

Закон управления:

\begin{equation}
u = k_1 x + k_2 r + u_a
\end{equation}

Алгоритмы адаптации:

\begin{equation}
\begin{aligned}
\dot{k}_1 &= -\gamma_1 e x \\
\dot{k}_2 &= -\gamma_2 e r \\
u_a &= -k \int_0^t e(\tau) d\tau
\end{aligned}
\end{equation}

\section{Адаптивные системы с неявной эталонной моделью}
\subsection{Общая структура систем с неявной эталонной моделью}
В системах с неявной эталонной моделью желаемое поведение задается неявно, через целевую функцию или функционал качества. Метод скоростного градиента применяется для минимизации этого функционала.

Общая структура:
\begin{itemize}
    \item Объект управления: $\dot{x} = f(x, u, \theta)$
    \item Целевая функция: $Q(x, u, \theta)$
    \item Регулятор: $u = u(x, \theta)$
    \item Адаптивный механизм: $\dot{\theta} = F(Q, x, u, \theta)$
\end{itemize}

\subsection{Алгоритмы с параметрической адаптацией}
Для систем с неявной моделью параметрическая адаптация основывается на градиентном спуске по целевой функции $Q(\theta)$:

\begin{equation}
\dot{\theta} = -\gamma \frac{\partial Q}{\partial \theta}
\end{equation}

\subsubsection{Пример параметрической адаптации в системе с неявной моделью}
Рассмотрим объект:

\begin{equation}
\dot{x} = ax + bu
\end{equation}

Целевая функция:

\begin{equation}
Q = \frac{1}{2}(x - r)^2
\end{equation}

Закон управления:

\begin{equation}
u = kx
\end{equation}

Алгоритм адаптации:

\begin{equation}
\dot{k} = -\gamma (x - r)x
\end{equation}

\subsection{Алгоритмы с сигнальной адаптацией}
При сигнальной адаптации в системах с неявной моделью управляющее воздействие формируется как:

\begin{equation}
u = -\gamma \frac{\partial Q}{\partial u}
\end{equation}

где $Q$ - целевая функция качества управления.

\subsubsection{Пример сигнальной адаптации в системе с неявной моделью}
Для того же объекта:

\begin{equation}
\dot{x} = ax + bu
\end{equation}

Целевая функция:

\begin{equation}
Q = \frac{1}{2}(x - r)^2
\end{equation}

Закон управления:

\begin{equation}
u = -\gamma b(x - r)
\end{equation}

\subsection{Алгоритмы с сигнально-параметрической адаптацией}
Сигнально-параметрическая адаптация в системах с неявной моделью комбинирует оба подхода:

\begin{equation}
\begin{aligned}
u &= u_{0}(\theta) - \gamma_{u} \frac{\partial Q}{\partial u} \\
\dot{\theta} &= -\gamma_{\theta} \frac{\partial Q}{\partial \theta}
\end{aligned}
\end{equation}

\subsubsection{Пример сигнально-параметрической адаптации в системе с неявной моделью}
Для объекта:

\begin{equation}
\dot{x} = ax + bu
\end{equation}

Целевая функция:

\begin{equation}
Q = \frac{1}{2}(x - r)^2
\end{equation}

Закон управления:

\begin{equation}
u = kx - \gamma b(x - r)
\end{equation}

Алгоритмы адаптации:

\begin{equation}
\begin{aligned}
\dot{k} &= -\gamma_k (x - r)x \\
\dot{\gamma} &= -\gamma_{\gamma} b(x - r)^2
\end{aligned}
\end{equation}

\end{document}