\documentclass[a4paper,14pt]{extarticle} % Используем extarticle для поддержки шрифта 14
\usepackage[utf8]{inputenc}
\usepackage[T2A]{fontenc}
\usepackage[russian]{babel}
\usepackage{amsmath}
\usepackage{amssymb}
\usepackage{graphicx}
\usepackage[left=3cm,right=2cm,top=2cm,bottom=2cm]{geometry}
\linespread{1.5}

\begin{document}

\begin{titlepage}
    \begin{center}
        \large
        Министерство науки и высшего образования Российской Федерации \\
        Санкт-Петербургский государственный электротехнический университет \\
        \textbf{«ЛЭТИ» им. В.И. Ульянова (Ленина)} \\
        Кафедра систем автоматического управления

        \vfill

        \textbf{Реферат} \\
        по дисциплине \\
        \textbf{«Нелинейное адаптивное управление в технических системах»}

        \vfill

        Студент группы 9492 \hfill Викторов А.Д. \\
        Преподаватель \hfill Путов В.В.

        \vfill
        Санкт-Петербург \\
        2024
    \end{center}
\end{titlepage}

\setcounter{page}{2}
\tableofcontents

\newpage
\section*{Введение}
Адаптивное и робастное управление занимают важное место в современной теории управления, поскольку позволяют обеспечивать устойчивость и качество работы нелинейных систем в условиях неопределенностей. Такие системы часто встречаются в реальных приложениях, включая робототехнику, авиацию и энергетику. Особое внимание уделяется задачам, где функциональные неопределенности и возмущения накладывают ограничения на синтез алгоритмов управления. В данной работе рассматриваются подходы к адаптивному и робастному управлению нелинейными системами в условиях функциональных и секторных ограничений на нелинейность.

\newpage
\section{Адаптивная стабилизация нелинейной системы с ограниченными функциональными неопределенностями}
\subsection*{Постановка задачи}
Рассматривается нелинейная система в виде:
\begin{equation}
\dot{x}(t) = f(x(t)) + b(x(t))u(t),
\end{equation}
где $x(t) \in \mathbb{R}^n$ — вектор состояния, $u(t) \in \mathbb{R}$ — управляющее воздействие, $f(x)$ и $b(x)$ — неизвестные функции. Функция $f(x)$ описывает динамику объекта и может содержать нелинейности, а $b(x)$ — коэффициент при управлении. 

Основное ограничение на $f(x)$ выражается в виде:
\begin{equation}
|f(x)| \leq \phi(|x|),
\end{equation}
где $\phi(|x|)$ — известная ограничивающая функция, а $b(x)$ удовлетворяет условию:
\begin{equation}
|b(x)| \geq b_{\text{min}} > 0.
\end{equation}

Цель управления — построить алгоритм $u(t)$, который стабилизирует состояние системы $x(t)$ к нулю.

\subsection*{Синтез алгоритма управления}
Для стабилизации используется подход, основанный на адаптивной оценке параметров $f(x)$. Закон управления имеет вид:
\begin{equation}
u(t) = -k(x) - \hat{f}(x),
\end{equation}
где $k(x)$ — линейный стабилизатор, а $\hat{f}(x)$ — оценка неизвестной функции $f(x)$.

Адаптация параметров выполняется с использованием следующего уравнения:
\begin{equation}
\dot{\hat{\theta}} = -\gamma e(t)\phi(x),
\end{equation}
где $e(t)$ — ошибка, $\phi(x)$ — базисная функция, $\gamma > 0$ — скорость адаптации. Этот алгоритм позволяет корректировать оценку $\hat{f}(x)$ в процессе управления.

Для доказательства устойчивости применяется метод Ляпунова с функцией:
\begin{equation}
V(x, \hat{\theta}) = \frac{1}{2}x^T x + \frac{1}{2\gamma} (\hat{\theta} - \theta)^2,
\end{equation}
где $\theta$ — истинное значение параметра.

\newpage
\section{Адаптивное управление нелинейной системой с ограниченными функциональными неопределенностями в условиях действия возмущающих воздействий}
\subsection*{Постановка задачи}
Модель системы дополняется внешним возмущением $d(t)$:
\begin{equation}
\dot{x}(t) = f(x(t)) + b(x(t))u(t) + d(t),
\end{equation}
где $d(t)$ удовлетворяет условию:
\begin{equation}
|d(t)| \leq d_{\text{max}}.
\end{equation}

Задача управления — разработать алгоритм, который стабилизирует систему и минимизирует влияние возмущений $d(t)$.

\subsection*{Синтез алгоритма адаптивного управления}
Закон управления состоит из двух частей:
\begin{equation}
u(t) = u_0(t) + u_d(t),
\end{equation}
где $u_0(t)$ — адаптивное управление для стабилизации, а $u_d(t)$ — корректирующий сигнал для компенсации возмущений.

Корректирующий сигнал определяется как:
\begin{equation}
u_d(t) = -k_d \text{sign}(d(t)),
\end{equation}
где $k_d$ — коэффициент компенсации.

Для адаптации параметров функции $f(x)$ используется модифицированный градиентный метод:
\begin{equation}
\dot{\hat{\theta}} = -\gamma e(t)\phi(x) + \lambda \hat{\theta},
\end{equation}
где $\lambda$ — коэффициент регуляризации.

Устойчивость системы подтверждается с использованием обобщенного критерия Ляпунова.

\section{Адаптивное и робастное управление по выходу нелинейными системами в условиях секторных ограничений на нелинейность}
\subsection*{Постановка задачи}
Система рассматривается с ограничениями на нелинейность в секторе:
\begin{equation}
k_1 y(t)^2 \leq y(t)u(t) \leq k_2 y(t)^2,
\end{equation}
где $k_1, k_2$ — коэффициенты, определяющие сектор ограничения.

Цель управления — обеспечить устойчивость системы и подавление неопределенностей.

\subsection*{Алгоритм управления}
Управление формируется в виде:
\begin{equation}
u(t) = -k_0 y(t) - \mu \text{sign}(y(t)),
\end{equation}
где $k_0$ обеспечивает стабилизацию, а $\mu$ подавляет неопределенности.

Коэффициенты выбираются из следующих условий:
\begin{equation}
k_0 > \frac{k_2}{b_{\text{min}}}, \quad \mu > d_{\text{max}}.
\end{equation}

\subsection*{Настройка коэффициентов регулятора}
Коэффициенты $k_0$ и $\mu$ подбираются с использованием численных методов, обеспечивая выполнение условий устойчивости и удовлетворение ограничениям.

\section*{Заключение}
Адаптивное и робастное управление в условиях ограничений на нелинейность позволяет эффективно стабилизировать системы при наличии неопределенностей и возмущений. В работе рассмотрены основные подходы к синтезу и настройке алгоритмов управления для различных классов задач.

\end{document}